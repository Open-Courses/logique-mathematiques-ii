\chapter{Mesures}

Nous supposons que nous travaillons dans un espace mesurable $(X, \mathcal{A})$.

%\begin{definition}
	%Soit $(X, \mathcal{A})$ un espace mesurable.
	%Soit $\mu : \mathcal{A} \rightarrow [-\infty, + \infty]$ une fonction.

	%On dit que $\mu$ est \textbf{additive} ou \textbf{finie additive} si pour
	%tout $A, B \in \mathcal{A}$,
	%\begin{equation}
		%\mu(A \union B) = \mu(A) + \mu(B)
	%\end{equation}
%\end{definition}

\begin{definition}
	Soit $(X, \mathcal{A})$ un espace mesurable.
	Soit $\mu : \mathcal{A} \rightarrow [0, + \infty]$ une fonction.

	On dit que $\mu$ est \textbf{mesure} si
	\begin{enumerate}
		\item $\mu(\emptyset) = 0$
		\item pour toute famille $\GSsequence{A}{n}{\naturel}$ d'éléments $2$ à
			$2$ disjoints de $\mathcal{A}$
			\begin{equation}
				\mu(\union_{n \in \naturel} A_{n}) = \sum_{n \in \naturel}
				\mu(A_{n})
			\end{equation}
	\end{enumerate}
\end{definition}

\begin{definition}
	Soit $(X, \mathcal{A})$ un espace mesurable.
	Soit $\mu : \mathcal{A} \rightarrow [0, + \infty]$ une fonction.

	On dit que $\mu$ est \textbf{une mesure (additive) finie} si $\mu$ est une
	mesure et si $\mu(X)$ est fini.
\end{definition}

Notre construction des hyperréels sera basée sur une mesure finie additive
respectant une propriété supplémentaire.

Donnons d'abord une première proposition.

\begin{proposition}
	Il existe des mesures additives finies $m : \powerSet{\naturel} \rightarrow
	\GSset{0, 1}$ tel que
	\begin{enumerate}
		\item $m$ est nul sur tout sous-ensemble fini de $\naturel$
		\item $m(\naturel) = 1$
	\end{enumerate}
\end{proposition}

\ifdefined\outputproof
\begin{proof}

\end{proof}
\fi

Remarquons que pour une telle mesure $m$, nous n'avons pas $m(A) = 1$ et
$m(\comp{A}) = 1$. En effet, si c'était le cas, on aurait $m(\naturel) = 2$. Or
$m(\naturel) = 1$.


