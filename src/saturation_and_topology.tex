\chapter{Saturation et topologie}

Dans ce chapitre, nous allons montrer que les notions d'analyse standard comme
la topologie, la continuité, les définitions d'ouverts, fermés, de compacts,
ainsi que la théorie des espaces métriques peuvent être étendues à l'analyse non
standard. Nous donnerons également des versions non standards de résultats
d'analyse standard comme le théorème d'Ascoli et le théorème d'Alaoglu.

Les hyperstructures et superstructures de $S$ ont été définies à partir de
familles basés sur $\naturel$ et un ultrafiltre non principal $\mathcal{U}$ sur
$\naturel$.

Par un même procédé, on peut définir les mêmes notions d'hyperstructures et
superstructures à partir d'une famille basée sur un ensemble infini quelconque
$I$ et un ultrafiltre $\mathcal{U}$ sur $I$.

Fixons un ensemble $S$ quelconque, $I$ un ensemble infini quelconque et
$\mathcal{U}$ un ultrafiltre non principal sur $I$. Construisons
$\hyperstructure{S}$ et $\superstructure{\hyperstructure{S}}$ depuis $S$, $I$ et
$\mathcal{U}$.

\begin{definition} [$\kappa$-saturé]
	Soit $\kappa$ un cardinal infini.
	On dit que $\superstructure{\hyperstructure{S}}$ est \textbf{$\kappa$-saturé} si pour tout
	ensemble $\Gamma$ tel que $\cardinal{\Gamma} < \kappa$, toute famille
	$\GSsequence{A}{\gamma}{\Gamma}$ est d'intersection non vide.

	On dit ausi que $\superstructure{\hyperstructure{S}}$ est \textbf{un modèle
	$\kappa$-saturé.}
\end{definition}

Nous avons un théorème important qui nous dit que nous pouvons toujours trouvé
un ultrafiltre $\mathcal{U}$ sur l'ensemble $I$ tel que le modèle
$\superstructure{\hyperstructure{S}}$ est $\kappa$-saturé pour tout $\kappa <
\cardinal{I}$.

\begin{theorem}
	Soit $\kappa$ un cardinal infini et soit $I$ un ensemble tel que
	$\cardinal{I} = \kappa^{+}$ où $\kappa^{+}$ est le successeur de $\kappa$.

	Soit $S$ un ensemble.

	Alors, il existe un ultrafiltre $\mathcal{U}$ sur $I$ tel que
	$\superstructure{\hyperstructure{S}}$ est $\kappa$-saturé.
\end{theorem}

Ce théorème est très intéressant car cela nous dit qu'étant donné un ensemble
$S$ et sa superstructure $\superstructure{S}$, nous pouvons construire un modèle
$\superstructure{\hyperstructure{S}}$ construit avec un ensemble $I$ tel que
$\cardinal{I} > \cardinal{\superstructure{S}}$ et un ultrafiltre sur $I$ qui
nous autorisera à prendre des familles indicées par $V(S)$ tout en pouvant
utiliser la saturation. Un tel modèle est appelé un modèle \textbf{polysaturé}.

\section{Topologie générale}

\label{general_topology}

Soit $(X, \tau)$ un espace topologique, et posons $S = \real \union X$.

Prenons un modèle polysaturé $\superstructure{\hyperstructure{S}}$.

\begin{definition} [Monade]
	Soit $a \in X$. On définit \textbf{la monade de $a$} comme le sous-ensemble
	$\monadOf{a}$ de $\hyperstructure{X}$ défini par:
	\begin{equation}
		\monadOf{a} = \inter \GSsetDef{\nonStandardVersion{\mathcal{O}} \subseteq
		\hyperstructure{X}}{a \in \mathcal{O} \wedge \mathcal{O} \in \tau}
	\end{equation}
\end{definition}

\begin{definition} [Proche d'un standard]
	Soit $\hyperelement{x} \in \hyperstructure{X}$. On dit que
	\textbf{$\hyperelement{x}$ est proche d'un
	standard} s'il existe $a \in X$ tel que $\hyperelement{x} \in \monadOf{a}$,
	c'est-à-dire qu'il existe $a \in A$ tel que $\hyperelement{x}$ est dans la
	monade de $a$. On dit aussi que \textbf{$\hyperelement{x}$ est proche du
	standard $a$}.

	L'ensemble de tous les points proches d'un standard est noté
	$\nearStandardSetOf{\hyperstructure{X}}$.\footnote{Pour
		\textit{nearstandard}.}
\end{definition}

La définition de la monade nous permet de caractériser avec des objets non
standards les ouverts, fermés et compacts de
de $X$.

\begin{proposition}
	Soit $A$ un sous-ensemble de $X$. Alors
	\begin{enumerate}
		\item $A$ est ouvert dans $(X, \tau)$ ssi pour tout $a \in A$,
			$\monadOf{a} \subseteq \nonStandardVersion{A}$.
		\item $A$ est fermé dans $(X, \tau)$ ssi quelque soit $\hyperelement{x}
			\in \nonStandardVersion{A}$ proche d'un standard $a \in X$, $a \in
			A$.
		\item $A$ est compact ssi tout $\hyperelement{x} \in
			\nonStandardVersion{A}$ est proche d'un standard $a \in A$.
	\end{enumerate}
\end{proposition}

\ifdefined\outputproof
\begin{proof}
	\begin{enumerate}
		\item $(1)$ Si $A$ est ouvert dans $(X, \tau)$, alors $\monadOf{a}
			\subseteq \nonStandardVersion{A}$ par définition de la monade de $a$.

			Supposons maintenant que $A$ n'est pas ouvert et montrons qu'il
			existe $a \in A$ tel qe $\monadOf{a} \inter \nonStandardVersion{A} \neq
			\emptyset$. Comme $A$ n'est pas ouvert, il existe $a \in A$ tel que
			pour tout ouvert $\mathcal{O}$ contenant $a$, $\mathcal{O} \inter
			\comp{A} \neq \emptyset$.

			Par saturation, en indiçant par les ouverts de $(X, \tau)$ la suite
			d'ensemble
			\begin{equation}
				A_{\mathcal{O}} := \GSsetDef{\nonStandardVersion{\mathcal{O}} \inter
				\nonStandardVersion{(\comp{A})}}{a \in \mathcal{O} \wedge
				\mathcal{O} \in \tau}
			\end{equation}
			on obtient
			\begin{equation}
				\inter_{\mathcal{O} \in \tau} A_{\mathcal{O}} \neq \emptyset
			\end{equation}
			Il est alors clair que tout élément $\hyperelement{x} \in
			\inter_{\mathcal{O} \in \tau} A_{\mathcal{O}}$ n'appartient pas à
			$A$ mais est bien dans $\monadOf{a}$.

		\item $(2)$ Soit $A$ compact et supposons que pour tout $a \in A$, il
			existe $x \in X$ et un voisinage ouvert $V_{a}$ contenant $a$ tel
			que $x \notin \hyperstructure{V_{a}}$.

			La fmaille $\GSsequence{V}{a}{A}$ étant un recouvrement de $A$ et
			$A$ étant compact, il existe un sous-recouvrement fini $V_{a_{1}},
			\cdots, V_{a_{n}}$ de $A$ ie $A \subseteq V_{a_{1}} \union \cdots
			\union V_{a_{n}}$. Comme le passage à la version non standard est
			stable par union finie, on a $\nonStandardVersion{A} \subseteq
			\hyperstructure{V_{a_{1}}} \union \cdots \union
			\hyperstructure{V_{a_{n}}}$. Contradiction avec le fait que $a$
			appartient à aucun des $\hyperstructure{V_{a_{i}}}$.
			\ldots
	\end{enumerate}
\end{proof}
\fi

Rappelons qu'un espace topologique $(X, \tau)$ est \textbf{Hausdorff} s'il est
séparable, c'est-à-dire que tout point $x, y \in X$, il existe des voisinages
ouverts $V_{x}$ et $V_{y}$ disjoints.

\begin{proposition}
	Soit $(X, \tau)$ un espace topologique de Hausdorff. Alors, tout point $x
	\in ns(\hyperstructure{X})$ est proche d'un unique non standard $a$.
\end{proposition}

\ifdefined\outputproof
\begin{proof}
	Soit $x \in \nearStandardSetOf{\hyperstructure{X}}$ ie il existe $a \in X$
	tel que $x \in \monadOf{a}$.

	S'il existait $b \in X$ tel que $x \in \mu(b)$, alors $a$ et $b$
	appartiennent aux mêmes ouverts. Or, il existe des ouverts les séparant.
\end{proof}
\fi

Vu l'unicité, on appelle cet unique $a$ la \textbf{partie
standard} de $x$, et est notée $\standardPart{x}$ ou $\standardPartExpo{x}$.

On peut ainsi définir la fonction
\begin{equation}
	st : \nearStandardSetOf{\hyperstructure{X}} \rightarrow X : x \rightarrow st(x)
\end{equation}

Cette notion étend bien la notion de non standard définie pour les hyperréels.

\begin{definition} [Infiniment proches]
	Soient $\hyperelement{x}, \hyperelement{y} \in
	\nearStandard{\hyperstructure{X}}$. On dit que $\hyperelement{x}$ et
	$\hyperelement{y}$ sont \textbf{infiniment proches}
	s'ils sont proches du même standard $a \in X$.

	On note $\hyperelement{x} \infinelyClosed \hyperelement{y}$.
\end{definition}

On obtient alors une autre formulation des équivalences données ci-dessous pour
les ouverts, fermés et compacts.

\begin{enumerate}
	\item $A$ est ouvert ssi $st^{-1}(A) \subseteq A$
	\item $A$ est fermé ssi $\nonStandardVersion{A} \inter
		\nearStandardSetOf{\hyperstructure{X}} \subseteq st^{-1}(A)$.
	\item $A$ est compact ssi $\nonStandardVersion{A} \subseteq st^{-1}(A)$.
\end{enumerate}

\begin{proposition}
	Soit $(X, \tau)$ un espace de Hausdorff et soit $A$ un sous ensemble interne
	de $\hyperstructure{X}$.

	Alors $st(A)$ est fermé dans $(X, \tau)$.
\end{proposition}

\ifdefined\outputproof
\begin{proof}
	Soit $a \in \adh{A}$. On a, pour tout voisinage de $a$ ouvert
	$\mathcal{O}_{a}$ , $\mathcal{O}_{a} \inter st(A) \neq \emptyset$.
	D'où cette propriété reste vraie par intersection finie. Par saturation, la
	famille $\GSsequence{B}{a}{A}$ définie par
	\begin{equation}
		B_{a} = \GSsetDef{\hyperstructure{\mathcal{O}_{a}} \inter
	A}{\mathcal{O}_{a} \in \tau}
	\end{equation}
	est d'intersection non vide.

	Soit $x$ un élément de l'intersection, en particulier. Alors, $x \in \monadOf{a}$ car $x \in
	\mathcal{O}_{a}$ pour tout $\mathcal{O}_{a} \in \tau$. D'où, $st(x) = a$ et
	$a \in st(A)$ car $x \in A$ en particulier.
\end{proof}
\fi

\begin{proposition}
	Soit $(X, \tau_{X})$ et $(Y, \tau_{Y})$ deux espaces topologiques. Soit $f :
	X \rightarrow Y$ et soit $a \in X$.
	Les assertions suivantes sont équivalentes:
	\begin{enumerate}
		\item $f$ est continue en $a$.
		\item $\nonStandardVersion{f}(\monadOf{a}) \subseteq \monadOf{f(a)}$
	\end{enumerate}
\end{proposition}

\ifdefined\outputproof
\begin{proof}
	$(1) \implies (2)$:

	Soit $x \in \monadOf{a}$. Montrons que $\nonStandardVersion{f}(x) \in
	\monadOf{f(a)}$, c'est-à-dire que $\nonStandardVersion{f}(x) \in
	\nonStandardVersion{\mathcal{O}_{f(a)}}$ pour tout ouvert
	$\mathcal{O}_{f(a)}$ contenant $f(a)$.
\end{proof}
\fi

\begin{proposition}
	Soit $(X, \tau_{X})$, $(Y, \tau_{Y})$ des espaces de Hausdorff tel que $(X,
	\tau_{X})$ est localement compact.

	Soit $f \in \continuousSpaceFunction{X}{Y}$ une fonction continue de $(X,
	\tau_{X})$ dans $(Y, \tau_{Y})$ et $\hyperelement{f} \in
	\nonStandardVersion{\continuousSpaceFunction{X}{Y}}$ un élément de la
	version non standard de $\continuousSpaceFunction{X}{Y}$.

	Alors, les assertions suivantes sont équivalentes:
	\begin{enumerate}
		\item $f = \standardPart{\hyperelement{f}}$
		\item quelque soit $x \in X$ proche d'un standard $a$,
			$\hyperelement{f}(\nonStandardVersion{x})$ est proche du standard
			$f(a)$.
	\end{enumerate}
\end{proposition}

\ifdefined\outputproof
\begin{proof}

\end{proof}
\fi

\begin{theorem} [de Tychonov]
	Soit $\GSsequence{X}{i}{I}$ une famille d'espaces compacts.

	Alors $\prod_{i \in I} X_{i}$ est compact.
\end{theorem}

\ifdefined\outputproof
\begin{proof}

\end{proof}
\fi

\begin{theorem} [d'Alaoglu]
	Soit $\GSnormedSpace{E}$ un espace de Banach et soit $\GSdual{E}$ son dual.

	Alors, la boule unité dans $\GSdual{E}$ est faiblement compacte.
\end{theorem}

\ifdefined\outputproof
\begin{proof}

\end{proof}
\fi

\begin{theorem} [d'Ascoli]
	Soit $\GSsequence{f}{i}{I}$ une famille de fonctions $f_{i} : X \rightarrow
	\real$ où $(X, \tau)$ est un espace compact. Supposons que
	$\GSsequence{f}{i}{I}$ est bornée et équicontinue.
\end{theorem}

\ifdefined\outputproof
\begin{proof}

\end{proof}
\fi

\section{Complétions, compactifications et structures non standard}

\label{completion_compactifitication_non_standard_hull}

Dans cette partie, nous allons étendre les structures d'espaces métriques à une
version non standard.

Dans le premier chapitre, nous avons défini les éléments finis de $\hyperreal$,
et nous avions noté cet ensemble $\finiteSet{\hyperreal}$. Les éléments finis
étaient définis comme les hyperréels bornés par la version non standard d'un
réel.
Nous allons généraliser cette notion pour un espace métrique quelconque.

Rappelons que nous travaillons dans $\superstructure{\hyperstructure{S}}$
polysaturé où nous avons posé $S$ comme l'union de chaque ensemble que nous
aurions besoin comme $X$, $\real$, etc. Cette hypothèse nous permet de parler des
versions non standard de n'importe quel élément construit sur $X$, $\real$, etc.
De plus, $\hyperstructure{S}$ est construit à partir d'un ensemble infini
$I$ quelconque et d'un ultrafiltre non principal sur celui-ci: nous travaillons
plus sur $\naturel$ !
Cependant, certaines preuves seront données dans le cas simplifié où $I =
\naturel$ et on supposera que l'ultrafiltre est non principal pour pouvoir
utiliser les cofinis.\footnote{Est-ce immédiat si nous sommes dans un modèle
	polysaturé ? En d'autres termes, est-ce qu'un ultrafiltre principal peut
engendrer un modèle polysaturé ?}

\subsection{Version non standard d'un espace métrique}

Dans la section précédente \ref{general_topology} sur la généralisation d'un
espace topologique à une version non standard, nous avons défini la monade d'un
élément de l'espace topologique $(X, \tau)$. Comme un espace métrique $(X, d)$
induit un espace topologique $(X, \tau_{d})$ où $\tau_{d}$ est la topologie
engendrée par les boules ouvertes, nous pouvons également parler de la monade.
Pour un élément $x \in X$, sa monade, notée $\monadOf{x}$, est
définie comme

\begin{align}
	\monadOf{x} & = \bigcap_{\mathcal{O}_{x} \in \tau_{d}}
	\nonStandardVersion{\mathcal{O}_{x}} \\
	& = \bigcap_{\epsilon > 0} \nonStandardVersion{B(x, \epsilon)} \subseteq
	\hyperstructure{X}
\end{align}

De plus, l'espace topologique induit $(X, \tau_{d})$ est de Hausdorff, donc tout
élément $\hyperelement{x} \in \hyperstructure{X}$ est dans la monade d'un seul
élément. Cet élément est appelé la partie standard de $\hyperelement{x} \in
\hyperstructure{X}$.

Tout élément $\hyperelement{x} \in \hyperstructure{X}$ appartenant à une monade
est dit \og proche d'un standard \fg, c'est-à-dire qu'il existe $a \in X$ tel que
$\hyperelement{x} \in \monadOf{a}$ ou encore qu'il existe $a \in X$ tel que
\begin{equation}
	\forall \epsilon > 0, \, \hyperelement{x} \in
	\nonStandardVersion{B(a, \epsilon)}
\end{equation}
Donc un élément est proche d'un standard, c'est-à-dire appartient à
$\nearStandard{\hyperstructure{X}}$, s'il est aussi proche qu'on veut d'un
élément $a \in X$.

Des équivalences dans le cas particulier où $I = \naturel$ sont données après.
Commençons d'abord par le cas général.

\begin{definition} [Elément fini]
	Soit $(X, d)$ un espace métrique et soit $\hyperelement{x} \in
	\hyperstructure{X}$.

	On dit que $\hyperelement{x}$ est \textbf{fini} s'il existe $a \in X$ tel que
	$\nonStandardVersion{d}(\hyperelement{x}, \nonStandardVersion{a}) \in
	\finiteSet{\hyperreal}$.

	L'ensemble des éléments finis est noté $\finiteSet{\hyperstructure{X}}$.
\end{definition}

En particulier, tout élément proche d'un standard est fini.

\begin{proposition}
	Soit $(X, d)$ un espace métrique.

	Alors $\nearStandard{\hyperstructure{X}} \subseteq
	\finiteSet{\hyperstructure{X}}$
\end{proposition}

\ifdefined\outputproof
\begin{proof}
	Evident.
\end{proof}
\fi

Remarquons que l'existence d'un $a \in X$ implique que c'est vrai pour tous
élément de $X$. En effet,

\begin{proposition}
	Soit $(X, d)$ un espace métrique et soit $\hyperelement{x} \in
	\hyperstructure{X}$. Les assertions suivantes sont équivalentes.

	\begin{enumerate}
		\item $\hyperelement{x}$ est fini.
		\item pour tout $a \in X$, $\nonStandardVersion{d}(\hyperelement{x},
			\nonStandardVersion{a}) \in \finiteSet{\hyperreal}$.
	\end{enumerate}
\end{proposition}

\ifdefined\outputproof
\begin{proof}
	Cela découle du fait que $d(x, b) \leq d(x, a) + d(a, b)$ pour tous $x, a, b
	\in X$ et que $d(a, b)$ est fini.
\end{proof}
\fi

Regardons un peu mieux la définition d'élément fini.
$\nonStandardVersion{d}(\hyperelement{x}, \nonStandardVersion{a})$ est un
hyperréel. S'il est fini, cela signifie qu'il existe $\epsilon > 0$ tel que
\begin{equation}
	\GSsetDef{i \in I}{\abs{d(x_{i}, a)} < \epsilon} \in \mathcal{U}
\end{equation}
Donc $\hyperelement{x} \in \hyperstructure{X}$ est fini si $\hyperelement{x}$ ne
s'éloigne pas trop souvent d'une boule $B(a, \epsilon)$, c'est-à-dire que
$\hyperelement{x}$ reste presque tout le temps dans une boule $B(a, \epsilon)$.

\begin{remarque}
	La définition d'élément fini sur un espace métrique étend bien la notion
	d'élément fini que nous avons étudiée dans le chapitre
	\ref{chapter_hyperreal} sur l'espace métrique réel usuel.
\end{remarque}

Sur l'ensemble des éléments finis $\finiteSet{\hyperstructure{X}}$, nous allons
créer une relation d'équivalence à partir de la distance.

\begin{proposition}
	Soit $(X, d)$ un espace métrique et soit $\finiteSet{\hyperstructure{X}}$ les
	éléments finis de $\hyperstructure{X}$.

	Alors la relation
	\begin{equation}
		\hyperelement{x} \simeq \hyperelement{y} \equivalence
		\nonStandardVersion{d}(\hyperelement{x}, \hyperelement{y})
		\infinelyClosed 0
	\end{equation}
	est une relation d'équivalence sur $\finiteSet{\hyperstructure{X}}$.
\end{proposition}

La relation d'équivalence nous dit que deux éléments sont en relation s'ils sont
infiniment proches relativement à la distance $d$. Remarquons que nous sommes
très proches de la relation d'équivalence définie dans le cas standard pour
quotienter l'espace des suites sur $X$ pour construire son complété.

\ifdefined\outputproof
\begin{proof}
	Rappelons que dans le cas hyperréel, $\hyperelement{x} \in \hyperreal$ est
	infiniment proche de $0$ si $\hyperelement{x}$ est un infinitésimal,
	c'est-à-dire que pour tout $\epsilon > 0$, $\nonStandardVersion{-\epsilon} <
	\hyperelement{x} < \nonStandardVersion{\epsilon}$.
	Nous devons donc montrer
	\begin{enumerate}
		\item pour tout $\hyperelement{x} \in \finiteSet{\hyperstructure{X}}$,
			$\nonStandardVersion{d}(\hyperelement{x}, \hyperelement{x})$ est
			infinitésimal.
		\item pour tout $\hyperelement{x}, \hyperelement{y} \in
			\finiteSet{\hyperstructure{X}}$,
			$\nonStandardVersion{d}(\hyperelement{x}, \hyperelement{y})$ est
			infinitésimal ssi
			$\nonStandardVersion{d}(\hyperelement{y}, \hyperelement{x})$ est
			infinitésimal.
		\item pour tout $\hyperelement{x}, \hyperelement{y}, \hyperelement{z}
			\in \finiteSet{\hyperstructure{X}}$,
			si $\nonStandardVersion{d}(\hyperelement{x}, \hyperelement{y})$ et
			$\nonStandardVersion{d}(\hyperelement{y}, \hyperelement{z})$ sont
			infinitésimaux, alors $\nonStandardVersion{d}(\hyperelement{x},
			\hyperelement{z})$ est inifinitésmal.
	\end{enumerate}
	Chaque point est évident par les propriétés sur la distance et de la
	définition de $\nonStandardVersion{d}$.
\end{proof}
\fi

Maintenant que nous avons une relation d'équivalence sur
$\finiteSet{\hyperstructure{X}}$, nous pouvons construire le quotient
$\quotientSet{\finiteSet{\hyperstructure{X}}}{\simeq}$ que nous notons
$\tilde{X}$.

\begin{remarque}
	$X$ s'identifie à $\quotientSet{X}{\simeq}$ car si $x, y \in X$ et $x \simeq
	y$, cela signifie que $\nonStandardVersion{d}(x, y) \infinelyClosed 0$,
	c'est-à-dire que $x = y$. On peut donc voir $X$ comme un sous-ensemble de
	$\quotientSet{\finiteSet{\hyperstructure{X}}}{\simeq}$.
\end{remarque}

Sur ce quotient, nous
allons définir naturellement une métrique $\tilde{d}$ à partir de la métrique $d$ sur $X$.

\begin{proposition}
	La fonction
	\begin{equation}
		\tilde{d} : \tilde{X} \cartprod \tilde{X} \rightarrow \real :
		(\tilde{x}, \tilde{y}) \rightarrow
		\standardPart{\hyperelementSequenceNoIndice{d}(\tilde{x}, \tilde{y})}
	\end{equation}
	est une métrique sur $\tilde{X}$.
\end{proposition}

\ifdefined\outputproof
\begin{proof}
	Vérifions d'abord que $\tilde{d}$ est bien définie.

	Soient $\hyperelement{x_{1}}$, $\hyperelement{x_{2}}$,
	$\hyperelement{y_{1}}$, $\hyperelement{y_{2}} \in
	\finiteSet{\hyperstructure{X}}$ tel que $\hyperelement{x_{1}} \simeq
	\hyperelement{x_{2}}$ et $\hyperelement{y_{1}} \simeq \hyperelement{y_{2}}$.
	Montrons que
	\begin{equation}
		st(\nonStandardVersion{d}(\hyperelement{x_{1}}, \hyperelement{y_{1}})) =
	st(\nonStandardVersion{d}(\hyperelement{x_{2}}, \hyperelement{y_{2}}))
	\end{equation}

	On a
	\begin{align}
		\nonStandardVersion{d}(\hyperelement{x_{1}}, \hyperelement{y_{1}})
		\leq & \nonStandardVersion{d}(\hyperelement{x_{1}},
		\hyperelement{x_{2}}) +
		\\
		& \nonStandardVersion{d}(\hyperelement{x_{2}}, \hyperelement{y_{2}}) +
		\\
		& \nonStandardVersion{d}(\hyperelement{y_{2}}, \hyperelement{y_{1}})
	\end{align}

	Comme $st : \finiteSet{\hyperreal} \rightarrow \real$ est un morphisme
	d'anneau et est une fonction croissante, on a
	\begin{align}
		st(\nonStandardVersion{d}(\hyperelement{x_{1}}, \hyperelement{y_{1}}))
		\leq & st(\nonStandardVersion{d}(\hyperelement{x_{1}},
		\hyperelement{x_{2}})) +
		\\
		& st(\nonStandardVersion{d}(\hyperelement{x_{2}}, \hyperelement{y_{2}})) +
		\\
		& st(\nonStandardVersion{d}(\hyperelement{y_{2}}, \hyperelement{y_{1}}))
	\end{align}

	Comme $\hyperelement{x_{1}} \simeq
	\hyperelement{x_{2}}$ et $\hyperelement{y_{1}} \simeq \hyperelement{y_{2}}$,
	on a $st(\nonStandardVersion{d}(\hyperelement{y_{2}}, \hyperelement{y_{1}}))
	= 0$ et \\ $st(\nonStandardVersion{d}(\hyperelement{x_{1}},
	\hyperelement{x_{2}})) = 0$.
	D'où
	\begin{equation}
		st(\nonStandardVersion{d}(\hyperelement{x_{1}}, \hyperelement{y_{1}})) \leq
	st(\nonStandardVersion{d}(\hyperelement{x_{2}}, \hyperelement{y_{2}}))
	\end{equation}
	On obtient l'autre inégalité par un même raisonnement.

	Vérifions maintenant que $\tilde{d}$ est une métrique sur $\tilde{X}$.

	\begin{enumerate}
		\item Soit $\tilde{x} \in \tilde{X}$, $\tilde{d}(\tilde{x}, \tilde{x}) =
			0$: vrai par la propriété $\nonStandardVersion{d}(\hyperelement{x},
			\hyperelement{x}) = 0$ où $\hyperelement{x}$ est un représentant de
			$\tilde{x}$.
		\item Soient $\tilde{x}, \tilde{y} \in \tilde{X}$, $\tilde{d}(\tilde{x},
			\tilde{y}) = \tilde{d}(\tilde{y}, \tilde{x})$: vrai par la
			propriété $\nonStandardVersion{d}(\hyperelement{x},
			\hyperelement{y}) = \nonStandardVersion{d}(\hyperelement{y},
			\hyperelement{x})$ où $\hyperelement{x}$ (resp. $\hyperelement{y}$)
			est un représentation de $\tilde{x}$ (resp. $\tilde{y}$).
		\item Soient $\tilde{x}$, $\tilde{y}$, $\tilde{z} \in \tilde{X}$,
			$\tilde{d}(\tilde{x}, \tilde{z}) \leq \tilde{d}(\tilde{x},
			\tilde{y}) + \tilde{d}(\tilde{y}, \tilde{z})$: vrai car $st$ est un
			morphisme d'anneau et une fonction croissante et
			$\nonStandardVersion{d}(\hyperelement{x}, \hyperelement{z}) \leq
			\nonStandardVersion{d}(\hyperelement{x}, \hyperelement{y}) +
			\nonStandardVersion{d}(\hyperelement{y}, \hyperelement{z})$ pour
			tout représentant.
	\end{enumerate}
\end{proof}
\fi

Nous obtenons alors un espace métrique $(\tilde{X}, \tilde{d})$. De plus, cet
espace métrique est complet.

\begin{proposition}
	$(\tilde{X}, \tilde{d})$ est un espace métrique complet.
\end{proposition}

\ifdefined\outputproof
\begin{proof}
	Soit $\GSsequence{\tilde{x}}{n}{\naturel}$ une suite de Cauchy dans
	$\tilde{X}$.

	Pour chaque $n \in \naturel$, prenons un représentant
	$\hyperelement{x_{n}}$ de $\tilde{x}_{n}$. On obtient donc une suite
	$\GSsequence{\hyperelement{x}}{n}{\naturel}$ dans
	$\finiteSet{\hyperstructure{X}}$.

	On étend cette suite $\GSsequence{\hyperelement{x}}{n}{\naturel}$ en une
	suite interne
	$(\hyperelement{x}_{\hyperelement{n}})_{\hyperelement{n} \in
	\nonStandardVersion{\naturel}}$.

	Comme $\GSsequence{\tilde{x}}{n}{\naturel}$ est de Cauchy,
	$\GSsequence{\hyperelement{x}}{n}{\naturel}$ l'est aussi.
	Alors, pour tout $m \in \naturel$, il existe $N_{m} \in \naturel$ tel que
	pour tout $\hyperelement{n} \in \nonStandardVersion{\naturel}$,
	$\hyperelement{n} \geq N_{m}$,
	\begin{equation}
		\nonStandardVersion{d}(\hyperelement{x}_{\nonStandardVersion{N_{m}}},
		\hyperelement{x}_{\hyperelement{n}}) \leq \frac{1}{m}
		\label{cauchy_internal_sequence}
	\end{equation}

	L'ensemble
	\begin{equation}
		B_{m} := \GSsetDef{\hyperelement{n} \in
		\nonStandardVersion{\naturel}}{\nonStandardVersion{d}(x_{\nonStandardVersion{N_{m}}},
		x_{\hyperelement{n}}) \leq \frac{1}{m}}
	\end{equation}
	possède des éléments finis arbitrairement grands. D'où, par overflow
	(\ref{overflow}), il existe un naturel non-standard infini $H_{m} \in
	B_{m}$.

	Posons
	\begin{equation}
		A_{m} := \GSsetDef{\hyperelement{k} \in
			\nonStandardVersion{\naturel}}{\nonStandardVersion{m} \leq
			\hyperelement{k} \leq H_{m}}
	\end{equation}
	La famille $\GSsequence{A}{m}{\naturel}$ possède la propriété
	d'intersection finie (vu que $H_{m}$ est infini pour tout $m \in \naturel$).

	Par saturation, il existe $H \in \nonStandardVersion{\naturel}$ tel que pour
	tout $m \in \naturel$, $\nonStandardVersion{m} \leq H$ (d'où $H$ est
	infini). En particulier (\ref{cauchy_internal_sequence}),
	$\nonStandardVersion{d}(\hyperelement{x_{H}},
	\hyperelement{x_{\nonStandardVersion{N_{m}}}})) \leq
	\frac{1}{m}$ pour tout $m \in \naturel$.

	Soit $m \in \naturel$, comme $\GSsequence{\hyperelement{x}}{n}{\naturel}$
	est de Cauchy, il existe $N_{m}$ tel que pour tout $n_{1}, n_{2} \in
	\naturel$,
	\begin{equation}
		\nonStandardVersion{d}(\hyperelement{x_{n_{1}}},
		\hyperelement{x_{n_{2}}}) \leq \frac{1}{m}
	\end{equation}

	Soit $n \in \naturel$ tel que $n \geq N_{m}$,
	par les propriétés sur $\nonStandardVersion{d}$,  et comme
	$\GSsequence{\hyperelement{x}}{\hyperelement{n}}{\nonStandardVersion{\naturel}}$ étend la
	suite $\GSsequence{\hyperelement{x}}{n}{\naturel}$, on a
	\begin{align}
		\nonStandardVersion{d}(\hyperelement{x_{n}}, \hyperelement{x_{H}}) &
		\leq \nonStandardVersion{d}(\hyperelement{x_{n}},
		\hyperelement{x_{N_{m}}}) +
		\nonStandardVersion{d}(\hyperelement{x_{N_{m}}}, \hyperelement{x_{H}}) \\
		& \leq \frac{1}{m} + \frac{1}{m}
	\end{align}

	d'où
	\begin{equation}
		st(\nonStandardVersion{d}(\hyperelement{x_{n}}, \hyperelement{x_{H}}))
		\leq \frac{2}{m}
	\end{equation}
	c'est-à-dire, en revenant dans $\tilde{X}$,
	\begin{equation}
		\tilde{d}(\tilde{x}_{n}, \tilde{x}_{H}) \leq \frac{2}{m}
	\end{equation}
	ou encore que $\GSsequence{\tilde{x}}{n}{\naturel}$ tend vers
	$\tilde{x}_{H}$.
	\label{finite_set_complete_metric}
\end{proof}
\fi

Souvent, l'espace métrique complet
$(\quotientSet{\finiteSet{\hyperstructure{X}}}{\simeq}, \tilde{d})$ est
trop gros pour être le complété de $(X, d)$. On a besoin de réduire l'ensemble
sur lequel on quotiente et c'est là qu'intervient
$\preNearStandard{\hyperstructure{X}}$ que nous allons définir.

La définition de \og proche d'un standard \fg est assez forte car elle demande
que l'élément soit presque tout le temps dans une boule centrée par un
\textit{même} point. Nous allons définir une notion un peu plus générale qui est
\og presque proche d'un standard \fg.

\begin{definition}
	Soit $(X, d)$ un espace métrique. Soit $\hyperelement{x} \in
	\hyperstructure{X}$.

	On dit que $\hyperelement{x}$ est \textbf{presque proche d'un standard} si
	pour tout $\epsilon > 0$, il existe $a \in X$ tel que $\hyperelement{x} \in
	\nonStandardVersion{B(a, \epsilon)}$ (c'est-à-dire
	$\nonStandardVersion{d}(\hyperelement{x}, \nonStandardVersion{a}) <
	\epsilon$).

	L'ensemble de tous les éléments presque proches d'un standard est noté
	$\preNearStandard{\hyperstructure{X}}$.
\end{definition}

Les presque proches d'un standard sont donc les éléments qui se trouvent presque
entièrement dans une boule centrée en un point $a \in X$, et ce quelque soit la
taille de la boule. En d'autres termes, ce sont les éléments qui sont toujours
au moins dans la boule d'un élément de $X$.

Regardons le lien avec les éléments proches d'un standard.

\begin{proposition}
	Soit $(X, d)$ un espace métrique.

	Alors $\nearStandard{\hyperstructure{X}} \subseteq
	\preNearStandard{\hyperstructure{X}}$.
	En d'autres termes, tout élément proche d'un standard est presque proche d'un
	standard.
\end{proposition}

\ifdefined\outputproof
\begin{proof}
	Soit $\hyperelement{x} \in \nearStandard{\hyperstructure{X}}$. Donc, il
	existe $a \in X$ tel que pour tout $\epsilon_{1} > 0$, $\hyperelement{x} \in
	\nonStandardVersion{B(a, \epsilon_{1})}$.

	Soit $\epsilon > 0$. En prenant le $a$ donné précédemment, on a bien
	$\nonStandardVersion{B(a, \epsilon)}$. D'où $\hyperelement{x} \in
	\preNearStandard{\hyperstructure{X}}$.
\end{proof}
\fi

Grâce à cette dernière proposition, nous obtenons la chaîne d'inclusion
\begin{equation}
	X \subseteq \nearStandard{\hyperstructure{X}} \subseteq
	\preNearStandard{\hyperstructure{X}} \subseteq
	\finiteSet{\hyperstructure{X}} \subseteq \hyperstructure{X}
\end{equation}

On peut alors parler de
$\quotientSet{\preNearStandard{\hyperstructure{X}}}{\simeq}$.

\begin{proposition}
	$\quotientSet{\preNearStandard{\hyperstructure{X}}}{\simeq}$ est fermé dans
	$\quotientSet{\finiteSet{\hyperstructure{X}}}{\simeq}$.

	En particulier, comme $\quotientSet{\finiteSet{\hyperstructure{X}}}{\simeq}$
	est complet, $\quotientSet{\preNearStandard{\hyperstructure{X}}}{\simeq}$ est
	complet.
\end{proposition}

\ifdefined\outputproof
\begin{proof}
	Soit $\GSsequence{\hyperelement{\tilde{x}}}{n}{\naturel}$ une suite
	d'élément de $\quotientSet{\preNearStandard{\hyperstructure{X}}}{\simeq}$
	convergeante vers $\hyperelement{\tilde{x}} \in
	\quotientSet{\finiteSet{\hyperstructure{X}}}{\simeq}$.

	Soit $\GSsequence{\hyperelement{x}}{n}{\naturel}$ une suite de
	$\preNearStandard{\hyperstructure{X}}$ construite en prenant des
	représentants de chaque $\hyperelement{\tilde{x_{n}}}$.
	Soit $\hyperelement{x}$ un représentant de $\hyperelement{\tilde{x}}$.

	Montrons que $\hyperelement{x} \in \preNearStandard{\hyperstructure{X}}$. On
	aura alors que $\hyperelement{\tilde{x}} \in
	\quotientSet{\preNearStandard{\hyperstructure{X}}}{\simeq}$.

	Il faut montrer que pour tout $\epsilon > 0$, il existe $a \in X$ tel que
	$\hyperelement{x} \in \nonStandardVersion{B}(a, \epsilon)$.

	Soit $\epsilon > 0$.
	\begin{itemize}
		\item On sait qu'il existe $n_{0} \in \naturel$ tel que pour tout $n
			\geq n_{0}$, $\nonStandardVersion{d}(\hyperelement{x},
			\hyperelement{x_{n}}) \leq \frac{\epsilon}{2}$
		\item il existe $a \in X$ tel que
			$\nonStandardVersion{d}(\hyperelement{x_{n_{0}}},
			\nonStandardVersion{a}) \leq \frac{\epsilon}{2}$.
	\end{itemize}

	D'où
	\begin{align}
		\nonStandardVersion{d}(\hyperelement{x}, \nonStandardVersion{a}) & \leq
		\nonStandardVersion{d}(\hyperelement{x}, \hyperelement{x_{n_{0}}}) +
		\nonStandardVersion{d}(\hyperelement{x_{n_{0}}}, \nonStandardVersion{a})
		\\
		& \leq \frac{\epsilon}{2} + \frac{\epsilon}{2}
	\end{align}
\end{proof}
\fi

\begin{proposition}
	$\quotientSet{\preNearStandard{\hyperstructure{X}}}{\simeq}$ est le complété
	de $(X, d)$.
\end{proposition}

\ifdefined\outputproof
\begin{proof}
	Il suffit de montrer que $X$ est dense dans
	$\quotientSet{\preNearStandard{\hyperstructure{X}}}{\simeq}$, c'est-à-dire
	$\adh{X} = \quotientSet{\preNearStandard{\hyperstructure{X}}}{\simeq}$.

	Clairement, comme
	$\quotientSet{\preNearStandard{\hyperstructure{X}}}{\simeq}$ est fermé, on a
	$\adh{X} \subseteq
	\quotientSet{\preNearStandard{\hyperstructure{X}}}{\simeq}$.
	%TODO
\end{proof}
\fi

\subsubsection*{Cas où $I = \naturel$}

Quand $I = \naturel$, nous obtenons des équivalences intéressantes et qui nous
permettent de catégoriser de manière \og standard \fg les proches d'un standards
et presque proches d'un standard. Ces
équivalences permettent de dire que la définition de
$\preNearStandard{\hyperstructure{X}}$ est plus générale que
$\nearStandard{\hyperstructure{X}}$.

\begin{proposition}
	\label{near_standard_subsequence_convergente}
	Soit $(X, d)$ un espace métrique et soit $\hyperelementSequence{x} \in
	\hyperstructure{X}$. Alors, les assertions suivantes sont équivalentes:

	\begin{enumerate}
	  \item $\hyperelementSequence{x} \in \nearStandard{\hyperstructure{X}}$
	  \item $\GSsequence{x}{n}{\naturel}$ possède une sous-suite convergente.
	\end{enumerate}
\end{proposition}

\ifdefined\outputproof
\begin{proof}
	$(1) \implies (2)$:

	Comme $\hyperelementSequence{x} \in \nearStandard{\hyperstructure{X}}$, il
	existe $a \in X$ tel que pour tout $\epsilon > 0$, $\hyperelementSequence{x}
	\in \nonStandardVersion{B(a, \epsilon)}$.

	On a donc
	\begin{equation}
		A_{\epsilon} := \GSsetDef{n \in \naturel}{x_{n} \in B(a, \epsilon)} \in
		\mathcal{U}
	\end{equation}
	et de plus, $A_{\epsilon}$ est infini car sinon $\comp{A_{\epsilon}}$
	est cofini et appartient donc à $\mathcal{U}$.
	En particulier, pour tout $m \in \naturel$, il existe $n_{m} \in A_{m}$ aussi grand
	qu'on veut tel que $x_{n_{m}} \in B(a, \frac{1}{m})$.

	On construit la sous-suite récursivement:
	\begin{itemize}
		\item Pour $m = 1$, on trouve un $n_{1}$ (par exemple $n_{1} =
			\min(A_{1})$) tel que $x_{n_{1}} \in B(a, 1)$.
		\item Pour $k = m + 1$, on prend un $n_{k} \in A_{k} \backslash \GSset{0, \dots,
				n_{m}}$ (pour être sûr que $n_{k} > n_{m}$) tel que $x_{n_{k}}
				\in B(a, \frac{1}{k})$.
	\end{itemize}

	La suite $(x_{n_{m}})_{m \in \naturel}$ ainsi construite converge vers $a$.

	$(2) \implies (1)$:

	Soit $\GSsequence{x'}{n}{\naturel}$ une sous suite de
$\GSsequence{x}{n}{\naturel}$ convergeant vers $a \in X$.

	Soit $\epsilon > 0$. Il existe $n_{0} \in \naturel$ tel que pour tout $n
	\geq  n_{0}$,
	\begin{equation}
	  d(x'_{n}, a) \leq \epsilon
	\end{equation}
	D'où, pour tout $\epsilon > 0$,
	\begin{equation}
	  A'_{\epsilon} := \GSsetDef{n \in \naturel}{d(x'_{n}, a) \leq \epsilon} \in
\mathcal{U}
	\end{equation}
	car
	\begin{equation}
		\overbrace{\GSsetDef{n \in \naturel}{n \geq n_{0}}}^{\subseteq
			A'_{\epsilon}}
	  \in \mathcal{U}
	\end{equation}

	D'où, $\hyperelementSequence{x'} \in \nonStandardVersion{B(a, \epsilon)}$.
	Il en est alors de même pour $\hyperelementSequence{x}$ car $A'_{\epsilon}
	\subseteq A_{\epsilon}$ où
	\begin{equation}
		A_{\epsilon} = \GSsetDef{n \in \naturel}{d(x_{n}, a) \leq \epsilon}
	\end{equation}
\end{proof}
\fi

\begin{proposition}
	Soit $(X, d)$ un espace métrique.

	Soit $\hyperelementSequence{x} \in \hyperstructure{X}$. Les assertions
suivantes sont équivalentes:

	\begin{enumerate}
		\item $\hyperelementSequence{x} \in
			\preNearStandard{\hyperstructure{X}}$
		\item $\GSsequence{x}{n}{\naturel}$ possède une sous-suite de Cauchy.
	\end{enumerate}
\end{proposition}

\ifdefined\outputproof
\begin{proof}
	$(1) \implies (2)$:

	Soit $m \in \naturel^{> 0}$. Comme $\hyperelementSequence{x} \in
	\preNearStandard{\hyperstructure{X}}$, on sait qu'il existe $a_{m} \in X$
	tel que
	\begin{equation}
		A_{m} = \GSsetDef{n \in \naturel}{d(x_{n}, a_{m}) \leq \frac{1}{m}} \in
		\mathcal{U}
	\end{equation}

	Posons
	\begin{equation}
		B_{m} = \inter_{1 \leq i \leq m} A_{i}
	\end{equation}
	Nous avons $B_{m} \in \mathcal{U}$ car chaque $A_{i} \in \mathcal{U}$.
	Nous avons aussi
	\begin{align}
		n \in B_{m} & \equivalence \forall i \in \GSset{1, \cdots, m}, d(x_{n},
		a_{i}) \leq \frac{1}{i} \\
		& \equivalence d(x_{n}, a_{1}) \leq 1 \\
		& \wedge d(x_{n}, a_{2}) \leq \frac{1}{2} \\
		& \wedge \cdots \\
		& \wedge d(x_{n}, a_{m}) \leq \frac{1}{m}
	\end{align}

	Nous remarquons que $B_{m}$ est infini et $m_{1} \leq m_{2} \equivalence
	B_{m_{2}} \subseteq B_{m_{1}}$.
	De plus, pour $j \leq m$, on a
	\begin{align}
		d(a_{j}, a_{m}) & \leq d(a_{j}, x_{m}) + d(x_{m}, a_{m}) \\
		& \leq \frac{1}{j} + \frac{1}{m} \\
		& \leq \frac{2}{j}
	\end{align} pour un $x_{m} \in B_{m}$.

	Passons à la construction de la sous-suite de Cauchy
	$\GSsequence{x'}{n}{\naturel}$. Celle-ci se fait de
	manière récursive comme lors de la preuve de
	\ref{near_standard_subsequence_convergente}.

	\begin{itemize}
		\item $n = 1$: $x'_{1} = x_{m_{1}}$ où $m_{1} = \min B_{1}$.
		\item $n = k + 1$: $x'_{n} = x_{m_{n}}$ où $m_{n} = \min B_{n}
			\backslash
			\GSset{0, \cdots, m_{k}}$.
	\end{itemize}

	Montrons maintenant que $(x_{m_{n}})_{n \in \naturel}$ est de Cauchy,
	c'est-à-dire que pour tout $m \in \naturel$, il existe $i \in \naturel$
	tel que pour tout $k \geq j \geq i \in \naturel$, $d(x_{m_{j}},
	x_{m_{k}}) \leq \frac{c}{m}$ où $c$ est une constante quelconque.

	Soit $m \in \naturel$. Posons $i = m$. Soit $k \geq j \geq
	i$. On a
	\begin{align}
		d(x_{m_{j}}, x_{m_{k}}) & \leq d(x_{m_{j}}, a_{j}) \\
		& + d(a_{j}, a_{k}) \\
		& + d(a_{k}, x_{m_{k}}) \\
	\end{align}
	et
	\begin{enumerate}
		\item $m_{j} \in B_{j} \implies d(x_{m_{j}}, a_{j}) \leq \frac{1}{j}$
		\item $j \leq k \implies d(a_{j}, a_{k}) \leq \frac{1}{j}$
		\item $m_{k} \in B_{k} \implies d(a_{k}, x_{m_{k}}) \leq \frac{1}{k}
			\leq \frac{1}{j}$
	\end{enumerate}

	Et comme $j \geq i = m$, cela nous donne
	\begin{equation}
		d(x_{m_{j}}, x_{m_{k}}) \leq \frac{3}{j} \leq \frac{3}{m}
	\end{equation}

	$(2) \implies (1)$:

	Soit $\GSsequence{x'}{n}{\naturel}$ une sous-suite de Cauchy de
	$\GSsequence{x}{n}{\naturel}$. Pour $\epsilon
	> 0$ fixé, il existe $n_{0}$ tel que pour tout $m, n \geq n_{0}$, $d(x'_{m},
	x'_{n}) \leq \epsilon$.

	En fixant $n = n_{0}$, on a
	\begin{equation}
		A'_{\epsilon} = \GSsetDef{m \in \naturel}{d(x'_{m}, x'_{n_{0}} \leq \epsilon} \in
		\mathcal{U}
	\end{equation}
	(car cofini). De plus,
	\begin{equation}
		A'_{\epsilon} \subseteq A_{\epsilon} = \GSsetDef{m \in
		\naturel}{d(x_{m}, x'_{n_{0}} \leq \epsilon}
	\end{equation}
	implique que $\nonStandardVersion{d}(\hyperelementSequence{x},
	\nonStandardVersion{x'_{n_{0}}}) \leq \epsilon$.

	D'où $\hyperelementSequence{x} \in \preNearStandard{\hyperstructure{X}}$.
\end{proof}
\fi

En particulier, nous avons l'égalité
entre $\nearStandard{\hyperstructure{X}} =
\preNearStandard{\hyperstructure{X}}$ si et seulement si $X$ est complet.

\begin{proposition}
	Soit $(X, d)$ un espace métrique. Alors, les assertions suivantes sont
	équivalentes.

	\begin{enumerate}
		\item $(X, d)$ est complet.
		\item $\preNearStandard{\hyperstructure{X}} =
			\nearStandard{\hyperstructure{X}}$
	\end{enumerate}
\end{proposition}

\ifdefined\outputproof
\begin{proof}
	$(1) \implies (2)$. Nous avons montré que $\nearStandard{\hyperstructure{X}}
	\subseteq \preNearStandard{\hyperstructure{X}}$. Il nous reste donc à
	montrer que $\preNearStandard{\hyperstructure{X}} \subseteq
	\nearStandard{\hyperstructure{X}}$.


	On a montré que toute suite
	$\GSsequence{x}{n}{\naturel}$ tel que $\hyperelementSequence{x} \in
	\preNearStandard{\hyperstructure{X}}$ contient une sous-suite de Cauchy
	$\GSsequence{x'}{n}{\naturel}$. Comme $X$ est complet, il existe
	alors un élément $x \in X$ tel que $\GSsequence{x'}{n}{\naturel}$
	converge vers $x$. Donc $\hyperelementSequence{x}$ contient une sous-suite
	convergeante, c'est-à-dire que $\hyperelementSequence{x} \in
	\nearStandard{\hyperstructure{X}}$;

	$(2) \implies (1)$.

	Soit $\GSsequence{x}{n}{\naturel}$ une suite de Cauchy de $X$ et notons
	$\hyperelementSequence{x}$ l'élément de $\hyperelement{X}$ engendré par
	$\GSsequence{x}{n}{\naturel}$. Comme $\hyperelementSequence{x}$ est de
	Cauchy, elle est dans $\preNearStandard{\hyperstructure{X}}$.

	Par hypothèse,
	$\preNearStandard{\hyperstructure{X}} \subseteq
	\nearStandard{\hyperstructure{X}}$ donc $\hyperelement{x} \in
	\nearStandard{\hyperstructure{X}}$.

	Comme $\hyperelementSequence{x} \in \nearStandard{\hyperstructure{X}}$,
	$\hyperelementSequence{x}$ contient une sous-suite convergente. Comme
	$\GSsequence{x}{n}{\naturel}$ est de Cauchy et qu'elle contient une suite
	convergente, $\GSsequence{x}{n}{\naturel}$ converge.
\end{proof}
\fi

\subsection{Espaces métriques non standards}

On peut également définir des espaces métriques non standards.
\begin{definition}
	\textbf{Un espace métrique non standard} est un ensemble interne $X$ muni
		d'une fonction interne $d : X \cartprod X \rightarrow \hyperreal$ tel
		que les axiomes de métriques sont respectées.
\end{definition}

Clairement, toute version non standard d'un espace métrique est un espace
métrique non standard.

On
peut généraliser les espaces $l_{p}(n)$ en leur version non standard en
remplaçant les suites par des suites internes.

Soient $p \in \hyperreal$ et $n \in \nonStandardVersion{\naturel}$.

On définit l'espace métrique $(l_{p}(n), d)$ où $l_{p}(n)$ est l'ensemble des
suites internes $x_{1}, \cdots, x_{n}$ d'hyperréels et
\begin{align} d(x, y) &
	:= \GSnormeDef{x - y}{p} \\ & := \sum_{i = 1}^{n}(x_{i} -
	y_{i})^{\frac{1}{p}}
\end{align}

On peut définir les éléments finis de $X$: on se fixe un point de référence $a
\in X$ et les éléments finis sont les $x \in X$ tel que $d(x, a)$ (qui est
hyperréel) est fini. L'ensemble des points finis est aussi noté $\finiteSet{X}$
et on définit de la même manière une relation d'équivalence $\simeq$ sur
$\finiteSet{X}$ ainsi que la métrique $\tilde{d}$ associée.
On obtient alors le même résultat, c'est-à-dire que
$(\quotientSet{\finiteSet{X}}{\simeq}, \tilde{d})$ est un espace complet.

\subsection{Espaces vectoriels internes}

Passons maintenant à la définition d'un espace vectoriel interne sur
$\hyperreal$, généralisation d'un espace vectoriel réel.

\begin{definition}
	\textbf{Un espace vectoriel interne sur $\hyperreal$} est un ensemble
	\textit{interne} $X$ muni d'une application \textit{interne} (addition)
	\begin{equation}
		X \cartprod X \rightarrow X : (x, y) \rightarrow x + y
	\end{equation}
	et d'une application \textit{interne} (multiplication scalaire)
	\begin{equation}
		\hyperreal \cartprod X \rightarrow X : (a, x) \rightarrow ax
	\end{equation}
	satisfaisant les axiomes usuelles d'espace vectoriel.

	$X$ est \textbf{de dimension $n$} s'il existe des éléments internes $x_{1},
	\cdots, x_{n} \in X$ tel que tout élément $x \in X$ s'écrit de manière
	unique comme combinaison linaire de $x_{1}, \cdots, x_{n}$ à coefficients
	dans $\hyperreal$.

	$X$ est \textbf{hyperfini dimensionnel} s'il est de dimension $N$ pour un $N
	\in \nonStandardVersion{\naturel}$.
\end{definition}


On obtient le résultat suivant directement en utilisant la polysaturation.

\begin{proposition}
	Soit $E$ un espace vectoriel de dimension infinie sur $\real$.

	Alors il existe un espace vectoriel interne de dimension hyperfinie sur
	$\hyperreal$, noté $X$ tel que $E \subseteq X \subseteq \hyperstructure{E}$.
\end{proposition}

\ifdefined\outputproof
\begin{proof}
	Soit $S \subseteq E$ un ensemble fini. On définit l'ensemble
	\begin{equation}
		A_{S} = \GSsetDef{X \subseteq \hyperstructure{E}}{X \text{ sous-espace
		linéaire de dimension hyperfinie avec } S \subseteq X}
	\end{equation}

	La famille $(A_{S})_{S \subseteq E, \text{fini}}$ possède la propriété des
	intersections finies non vides donc par polysaturation, il existe $X \in
	\inter_{S \subseteq E, \text{fini}} A_{S}$ qui satisfait la proposition.
\end{proof}
\fi
