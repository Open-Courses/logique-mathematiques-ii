%        File: Logique.Mathematique.II.tex
%     Created: Die Okt 06 01:00  2015 C
% Last Change: Die Okt 06 01:00  2015 C
%
\documentclass[a4paper, 12pt]{report}

\usepackage[french]{babel}
\usepackage[T1]{fontenc}
\usepackage[utf8]{inputenc}
\usepackage{hyperref}

\usepackage{amsmath}
\usepackage{amsthm}
\usepackage{amssymb}
\usepackage{physics}
\usepackage[all]{xy}

\input{../latex-macros-math/macros_math}
\usepackage{amsfonts}
\usepackage{amssymb}
\usepackage{amsmath}
\usepackage{amsthm}
\usepackage{mathrsfs}

\newtheorem{definition}{Définition}[chapter]

\newtheorem{proposition}[definition]{Proposition}
\newtheorem{lemma}[definition]{Lemme}
\newtheorem{corollary}[definition]{Corollaire}
\newtheorem{theorem}[definition]{Théorème}

\newtheorem*{exemple}{Exemple}
\newtheorem*{question}{Questions}
\newtheorem*{remarque}{Remarque}
\newtheorem*{notation}{Notation}

\newtheorem{exercice}{Exercice}[chapter]


\title{Logique mathématique II}
\author{Danny Willems}

\begin{document}

\maketitle

\tableofcontents

\chapter{Analyse non standard et hyperréels}

\section{Construction}

Nous allons construire l'ensemble des hyperréels qui étendra le corps des réels.
Nous définirons ensuite les ensembles internes, les éléments infinitésimaux et
les éléments finis. Nous pourrons donner une définition précise de la notion
d'infénitésimal à travers les éléments infinitésimaux et nous verrons le lien
avec les suites.

On pourra remarquer que la notion d'hyperréel peut être étendue à n'importe quel
ensemble $S$. On parlera alors de l'hyperstructure de $S$.

Soit $\real$ l'ensemble des réels. Soit $\mathcal{U}$ un ultrafiltre non
principal sur $\naturel$.

On définit la relation d'équivalence $\sim$ sur $\real^{\naturel}$ tel que
\begin{equation}
	\GSsequence{x}{n}{\naturel} \sim \GSsequence{y}{n}{\naturel} \equivalence \GSsetDef{n \in \naturel}{x_{n} = y_{n}} \in \mathcal{U}
\end{equation}

On obtient alors \textbf{l'ultrapuissance de $\real$} noté $\real^{\mathcal{U}}$,
ou encore $\hyperreal$. Cet ultrapuissance est appelé \textbf{l'ensemble des
hyperréels}.

On note les éléments de $\hyperreal$ par $\hyperelement{x}$,
$[x_{n}]$ ou tout simplement $x$ quand le contexte est clair. Nous ferons la
distinction quand cela est nécessaire.

On peut étendre les opérations et l'ordre usuels de $\real$ à $\hyperreal$ de la façon
suivante:

Soient $\hyperelement{x}$ et $\hyperelement{y}$
\begin{enumerate}
	\item $\hyperelement{x} +_{\hyperreal} \hyperelement{y} = [(x_{n} +
		y_{n})_{n \in \naturel}]_{\mathcal{U}}$
	\item $\hyperelement{x} ._{\hyperreal} \hyperelement{y} = [(x_{n} .
		y_{n})_{n \in \naturel}]_{\mathcal{U}}$
	\item $\hyperelement{x} <_{\hyperreal} \hyperelement{y} \equivalence
		\GSsetDef{n \in \naturel}{x_{n} < y_{n}} \in \mathcal{U}$
\end{enumerate}

On obtient évidemment une injection de $\real$ dans $\hyperreal$ avec
\begin{equation}
	i : \real \rightarrow \hyperreal : x \rightarrow [(x, x, \dots)]_{\mathcal{U}}
\end{equation}

où on prend la classe d'équivalence de la suite constante de valeur $x$. Nous
garderons la notation $i(x)$ pour désigner l'hyperréel défini à partir du réel
$x$.

Ces deux opérations et l'ordre définies étendent les opérations réelles et
donnent une structure de corps ordonné sur $\hyperreal$.

\begin{proposition}
	$(\hyperreal, +_{\hyperreal}, ._{\hyperreal}, i(0), i(1),
<_{\hyperreal})$ est un corps ordonné.
\end{proposition}

\section{Eléments infinitésimaux et finis}

Maintenant que nous avons construit les hyperréels, nous allons distinguer trois
types d'éléments. Nous souhaitons définir \textit{être proche de zéro} ie
\textit{infinitésimal}, \textit{être proche d'un réel} et \textit{être infini}.

\begin{definition} [Infinitésimal]
	Soit $[x_{n}] \in \hyperreal$ et soit $i : \real \rightarrow \hyperreal$
	l'injecion de $\real$ dans $\hyperreal$.

	On dit que $[x_{n}]$ est \textbf{infinitésimal} si pour tout $a \in \real^{>
	0}$,
	\begin{equation}
		i(-a) \leq [x_{n}] \leq i(a)
	\end{equation}
\end{definition}

\begin{exemple}
	$i(0)$, $[\frac{1}{n}]$ et $[\frac{1}{\sqrt{n}}]$ sont des infinitésimaux.
\end{exemple}

\begin{definition} [Fini et infini]
	Soit $[x_{n}] \in \hyperreal$ et soit $i : \real \rightarrow \hyperreal$
	l'injecion de $\real$ dans $\hyperreal$.

	On dit que $[x_{n}]$ est \textbf{fini} si il existe $a \in \real^{> 0}$,
	\begin{equation}
		i(-a) \leq [x_{n}] \leq i(a)
	\end{equation}

	Sinon, $[x_{n}]$ est dit \textbf{infini}.
\end{definition}

\begin{exemple}
	Tous les hyperréels $i(x)$ sont finis. Les hyperréels $[n]$, $[n^{2}]$ sont
	infinis.
\end{exemple}

\begin{proposition}
	Soient $[x], [y] \in \hyperreal$. Alors

	\begin{enumerate}
		\item Si $[x]$ et $[y]$ sont infinitésimaux, alors $[x] + [y]$ est
			infinitésimal.
		\item Si $[x]$ et $[y]$ sont finis, alors $[x] + [y]$ est fini.
		%\item Si $[x]$ et $[y]$ sont infinis, alors $[x] + [y]$ est infini.
		\item Si $[x]$ est fini et $[y]$ est infini, alors $[x] [y]$ est infini.
	\end{enumerate}
\end{proposition}

\ifdefined\outputproof
\begin{proof}
	Evident.
\end{proof}
\fi

\begin{definition} [Infiniment proches]
	Soient $[x]$, $[y] \in \hyperreal$.

	On dit que $[x]$ et $[y]$ sont \textbf{infiniment proches} si $[x] - [y]$ est
	infinitésimal.

	On note alors $[x] \approx [y]$.
\end{definition}

\begin{proposition}
	Soit $[x] \in \hyperreal$ un hyperréel fini.

	Alors, il existe un unique $a \in \real$ et un unique $[\epsilon] \in \hyperreal$
	infinitésimal tel que
	\begin{equation}
		[x] = i(a) + [\epsilon]
	\end{equation}

	En particulier, il existe un unique réel $a$ tel que $[x] \approx i(a)$.
\end{proposition}

\ifdefined\outputproof
\begin{proof}

\end{proof}
\fi

\begin{definition} [Partie standard]
	Soit $[x] \in \hyperreal$ fini.

	L'unique réel $a$ tel que $i(a) \approx [x]$
	est appelé \textbf{partie standard de $[x]$} et est noté
	$\standardPartExpo{x}$
	ou encore $\standardPart{x}$.
\end{definition}

\begin{notation}
	Pour $A \subseteq \hyperreal$ un sous-ensemble d'hyperréel fini, on note
	\begin{equation}
		\standardPart{A} = \GSsetDef{st(x)}{x \in A}
	\end{equation}
\end{notation}

Montrons quelques applications des infinitésimaux et plus généralement des
notions que nous avons vues.

\begin{proposition}
	Soit $\GSsequence{x}{n}{\naturel}$ une suite de réels convergeant vers un
	réel $x \in \real$.

	Alors, $\hyperelement{x} \approx i(x)$.
\end{proposition}

\ifdefined\outputproof
\begin{proof}
	Nous devons montrer que pour tout $a \in \real^{> 0}$, on a
	\begin{equation}
		i(x - a) < \hyperelement{x} < i(x + a)
	\end{equation}
	Pour un $\epsilon > 0$ fixé, l'ensemble
	\begin{equation}
		\GSsetDef{n \in \naturel}{x - \epsilon < x_{n} < x + \epsilon}
	\end{equation}
	est cofini, donc est dans l'ultrafiltre $\mathcal{U}$.
\end{proof}
\fi

\section{Ensembles et fonctions internes}

\begin{definition} [Ensembles internes et externes]
	Soient $\GSsequence{A}{n}{\naturel}$ une suite de sous-ensembles de $\real$.

	On définit le sous ensemble hyperréel $\spanspace{A_{n}}$
	tel que
	\begin{equation}
		\hyperelement{x} \in \spanspace{A_{n}} \equivalence \GSsetDef{n \in
		\naturel}{x_{n} \in A_{n}} \in \mathcal{U}
	\end{equation}

	En d'autre termes, pour appartenir à $\spanspace{A_{n}}$, il faut que les
	éléments $x_{n}$ apparaissent \textit{assez souvent} dans les fibres $A_{n}$.

	Les ensembles d'hyperréels pouvant être obtenus de cette façon sont appelés
	\textbf{ensembles internes}. Les autres sont appelés \textbf{externes}.

	On note l'ensemble des ensembles internes par $\internalSet{\hyperreal}$

	% Besoin d'étendre la notion aux suites d'ordinaux pour définir la
	% cardinalité \ldots
	%On définit alors \textbf{la cardinalité interne de $\spanspace{A_{n}}$ comme
	%l'élément \ldots
\end{definition}

\begin{exemple}
	Soient $[a]$ et $[b]$ deux hyperréels. Alors l'ensemble
	\begin{equation}
		[ [a], [b] ] = \GSsetDef{[x] \in \hyperreal}{[a] \leq [x] \leq [b]}
	\end{equation}
	est interne et est obtenu en prenant $A_{n} = [a_{n}, b_{n}]$.
\end{exemple}

\begin{definition} [Fonctions internes]
	Soient $\GSsequence{f}{n}{\naturel}$ une suite de fonctions de $\real$ dans
	$\real$.

	On définit la fonction $\spanspace{f_{n}} : \hyperreal \rightarrow
	\hyperreal$ tel que pour tous hyperréels $\hyperelement{x}$
	\begin{equation}
		\spanspace{f_{n}}(\hyperelement{x}) = [f_{n}(x_{n})_{n \in
		\naturel}]_{\mathcal{U}}
	\end{equation}

	Les fonctions $f : \hyperreal \rightarrow \hyperreal$ pouvent être obtenues
	de cette façon sont appelées \textbf{fonctions internes}.
\end{definition}

\begin{proposition}
	Soient $\spanspace{A_{n}}$ et $\spanspace{B_{n}}$ deux ensembles internes,
	alors
	\begin{enumerate}
		\item $\spanspace{A_{n}} \inter \spanspace{B_{n}} = \spanspace{A_{n}
			\inter B_{n}}$
		\item $\spanspace{A_{n}} \union \spanspace{B_{n}} = \spanspace{A_{n}
			\union B_{n}}$
		\item $\comp{\spanspace{A_{n}}} = \spanspace{\comp{A_{n}}}$
	\end{enumerate}

	De manière générale, cela est vrai pour n'importe quelle intersection et
	union finie.
\end{proposition}

\ifdefined\outputproof
\begin{proof}
	Cela découle des propriétés sur les ultrafiltres.
\end{proof}
\fi

\begin{proposition}
	Soit $(A^{i})_{i \in \naturel}$ une suite d'ensembles internes de
	$\hyperreal$.
	Chaque $A^{i}$ est de la forme $\spanspace{A^{i}_{n}}$.

	Si pour tous $i \in \naturel$,
	\begin{equation}
		A^{i} \neq \emptyset
	\end{equation}
	alors il existe $i \in \naturel$ tel que pour tous $n \in \naturel$,
	\begin{equation}
		A^{i}_{n} \neq \emptyset
	\end{equation}
\end{proposition}

\ifdefined\outputproof
\begin{proof}
	TODO
	%Si pour tous $i \in \naturel$, il existe $n \in \naturel$ tel que $A^{i}_{n
	%\in \naturel}$, on peut alors construire $\spanspace{A^{n}_{i}}$ qui est
	%vide car pour tous $i \in \naturel$, $A^{n}
\end{proof}
\fi

\begin{proposition}
	Soit $\spanspace{A} \subseteq \hyperreal$ un ensemble interne non vide.

	Si $\spanspace{A}$ admet un majorant alors $\spanspace{A}$ admet un supremum.
\end{proposition}

\begin{proof}
	Soit $\spanspace{A} = \spanspace{A_{n}}$ borné par $[a_{n}] \in \hyperreal$
	c'est-à-dire que $\spanspace{A_{n}} < [a_{n}]$ ie
	\begin{equation}
		\overbrace{\GSsetDef{n \in \naturel}{A_{n} < a_{n}}}^{= I} \in \mathcal{U}
	\end{equation}
	Pour tous $i \in I$, on a $\sup(A_{i})$ qui existe.
	Posons $x_{i} = \sup(A_{i})$ si $i \in I$ et $0$ sinon.

	En particulier, $x_{i}$ est bien un majorant de $\spanspace{A_{n}}$ et c'est
	le plus petit. En effet, si $[y_{n}]$ majore $\spanspace{A_{n}}$, notons $J$
	l'ensemble
	\begin{equation}
		\GSsetDef{n \in \naturel}{A_{n} < y_{n}} \in \mathcal{U}
	\end{equation}
	En particulier, on a $I \inter J \in \mathcal{U}$ et on a bien $[x_{i}] <
	[y_{i}]$ car
	\begin{equation}
		I \inter J \subseteq \GSsetDef{n \in \naturel}{x_{i} < y_{i}}
	\end{equation}
	D'où,
	\begin{equation}
		\GSsetDef{n \in \naturel}{x_{i} < y_{i}} \in \mathcal{U}
	\end{equation}
	Donc $[x_{i}] < [y_{i}]$.
\end{proof}

On vient donc de montrer que le principe du supremum reste valable pour les
\textit{ensembles internes} dans le cas hyperréels.

Ce principe permet de distinguer les ensembles internes et externes. Si un
ensemble $\spanspace{A_{n}}$ n'a pas de borne supérieure, alors c'est un
ensemble externe.

\begin{proposition}
	Soit $A \subseteq \hyperreal$ un ensemble interne.

	\begin{enumerate}
		\item Si $A$ contient des élément finis arbitrairement grands, alors $A$
			contient un élément infinis.
		\item Si $A$ contient des éléments positifs infinis arbitrairement
			petits, alors $A$ contient un élément fini.
	\end{enumerate}
\end{proposition}

\ifdefined\outputproof
\begin{proof}
	\begin{enumerate}
		\item
		\item
	\end{enumerate}
\end{proof}
\fi

\begin{theorem} [$\aleph_{1}$-saturation]
	\label{theorem:aleph_1_saturation}
	Soit $(A^{i})_{i \in \naturel}$ une suite d'ensembles internes de
	$\hyperreal$ tel que pour tous $I \in \naturel$
	\begin{equation}
		\inter_{i \leq I} A^{i} \neq \emptyset
	\end{equation}
	alors
	\begin{equation}
		\inter_{n \in \naturel} A^{n} \neq \emptyset
	\end{equation}
\end{theorem}

Remarquons que cela n'est pas nécessairement vrai que cela reviendrait à dire
que l'intersection dénombrable est stable dans un ultrafiltre, ce qui n'est pas
nécéssairement vrai.

\ifdefined\outputproof
\begin{proof}
	Les $A^{i}$ sont de la forme $\spanspace{A^{i}_{n}}$. Comme
	\begin{equation}
		\inter_{i \leq I} A^{i} \neq \emptyset
	\end{equation}
	pour tous $I \in \naturel$, les $A^{i} \neq
	\emptyset$. En particulier, on peut supposer que $A^{0}_{n}$ est non vide
	pour tous $n \in \naturel$.

	On a, pour tous $I \in \naturel$,
	\begin{equation}
		\inter_{i \leq I} \spanspace{A^{i}_{n}} = \spanspace{\inter_{i \leq I}
		A^{i}_{n}}
	\end{equation}
	ce qui implique
	\begin{equation}
		\GSsetDef{n \in \naturel}{\inter_{i \in I} A^{i}_{n} \neq \emptyset} \in
		\mathcal{U}
	\end{equation}

	Soit $n \in \naturel$. Posons
	\begin{equation}
		I_{n} = \max \overbrace{\GSsetDef{I \in \naturel}{\inter_{i \leq I}
		A^{i}_{n}} \neq \emptyset \wedge I \leq n}^{= X_{n}}
	\end{equation}

	Comme on a supposé que $A^{0}_{n}$ est non vide pour tous $n \in \naturel$,
	on a $I_{n} \geq 0$ car l'ensemble est non vide ($0$ appartient pour tous $n
	\in \naturel$) et le max existe car $I \leq n$.

	Pour $n$ fixé, on prend un élément $x_{n} \in \inter_{i \in I_{n}}
	A^{i}_{n}$. On construit ainsi une suite $(x_{n})_{n \in \naturel}$.

	Montrons maintenant que $[x_{n}] \in A^{I}$ pour tous $I \in \naturel$. Pour
	un $I \in \naturel$ fixé, on a
	\begin{equation}
		\GSsetDef{n \in \naturel}{I \leq I_{n}} \subseteq \GSsetDef{n \in
		\naturel}{x_{n} \in A^{I}_{n}}
	\end{equation}
	car si $I \leq I_{n}$, on a $I \in X_{n}$ et par conséquence, comme $x_{n}
	\in \inter_{i \in I_{n}} A^{i}_{n}$ et $\inter_{i \in I} A^{i}_{n} \subseteq \inter_{i
		\in I_{n}} A^{i}_{n}$, $n \in \GSsetDef{m \in \naturel}{x_{m} \in
		A^{I}_{m}}$.

	De plus, par définition de $I_{n}$.
	\begin{equation}
		\GSsetDef{n \in \naturel}{I \leq I_{n}} = \GSsetDef{n \in \naturel}{I
		\leq n} \inter \GSsetDef{n \in \naturel}{\inter_{i \leq I} A^{i}_{n}
	\neq \emptyset}
	\end{equation}
	Comme $\GSsetDef{n \in \naturel}{I \leq n}$ est cofini, il appartient à
	$\mathcal{U}$. D'où $\GSsetDef{n \in \naturel}{I \leq I_{n}} \in
	\mathcal{U}$ et par conséquence,
	\begin{equation}
		\GSsetDef{n \in \naturel}{x_{n} \in A^{I}_{n}} \in \mathcal{U}
	\end{equation}
	ie $[x_{n}] \in A^{I}$.
\end{proof}
\fi

\begin{corollary}
	Soit $\spanspace{A^{i}}_{i \in \naturel}$ une suite d'ensembles internes de
	$\hyperreal$.
	Alors, LASSE
	\begin{enumerate}
		\item $\union_{i \in \naturel} A^{i}$ est interne
		\item il existe $n \in \naturel$ tel que
			\begin{equation}
				\union_{i \in \naturel} A^{i} = \union_{i \leq n} A^{i}
			\end{equation}
	\end{enumerate}
\end{corollary}

\ifdefined\outputproof
\begin{proof}
	$(2) \implies (1)$ est évident.

	$(1) \implies (2)$ Supposons que $\union_{n \in \naturel A_{n}}$ est
	interne et posons $A = \union_{n \in \naturel A_{n}}$. Alors, pour tout $n
	\in \naturel$, $A/A_{n}$ est aussi interne.
	De plus,
	\begin{equation}
		\inter_{n \in \naturel} A/A_{n} = \emptyset
	\end{equation}
	Donc, par la contraposée de la $\aleph_{1}$-saturation
	(\ref{theorem:aleph_1_saturation}), il existe $N \in \naturel$ tel que
	\begin{equation}
		\inter_{n \leq N} A/A_{n} = \emptyset
	\end{equation}
	c'est-à-dire
	\begin{equation}
		A = \union_{n \leq N} A_{n}
	\end{equation}
\end{proof}
\fi

\begin{proposition}
	Soit $A \subseteq \hyperreal$ un sous-ensemble interne.

	Alors, $st(A)$ est fermé dans $(\real, \tau_{\abs{.}})$.
\end{proposition}

\ifdefined\outputproof
\begin{proof}

\end{proof}
\fi

\section{Ensembles standards et hyperfinis}

Pour les hyperréels $[x]$, on a distingués ceux qui étaient finis. On a alors
défini une fonction $st : Fin(\hyperreal) \rightarrow \real$ qu'on a appelé la
partie standard d'un élément fini.
Nous allons définir la même chose, mais cette fois-ci avec les fonctions et les
ensembles.

\begin{definition}
	Soit $A \subseteq \real$ un sous-ensemble réel. On définit \textbf{la
	version non standard de $A$} comme l'ensemble interne $\nonStandard{A} =
	\spanspace{A_{n}}$ où pour tous $n \in \naturel$, $A_{n} = A$.

	On dit qu'un sous-ensemble de $\hyperreal$ est \textbf{standard} s'il
	est de la forme $\nonStandard{A}$.
\end{definition}

\begin{definition}
	Soit $f : \real \rightarrow \real$ une fonction. On définit \textbf{la
	version non standard de $f$} comme la fonction interne
	\begin{equation}
		\nonStandard{f} = \spanspace{f_{n}} : \hyperreal \rightarrow \hyperreal
	\end{equation}

	où pour tous $n \in \naturel$, $f_{n} = n$.

	On dit qu'une fonction $\spanspace{f}$ est \textbf{standard} si elle est
	de la forme $\nonStandard{f}$.
\end{definition}

%\begin{remarque}
	%Pour un ensemble $A \subseteq \real$, sa version non standard
	%$\nonStandard{A}$ contient bien plus d'élément que \textit{simplement} que
	%$i(A)$.
%\end{remarque}

\begin{proposition}
	Soit $A \subseteq \real$.

	Alors $i(A) \subseteq \nonStandard{A}$.

	De plus, $A$ est fini si et seulement si $i(A) = \nonStandard{A}$.
\end{proposition}

\ifdefined\outputproof
\begin{proof}

\end{proof}
\fi

Remarquons que la version non standard $\nonStandard{f}$ d'une fonction $f :
\real \rightarrow \real$ étend la fonction $f$ dans le sens que pour tous $a \in
\real$:

\begin{equation}
	\nonStandard{f}(i(a)) = \spanspace{f}(i(a)) = \spanspace{f(a)} = i(f(a))
\end{equation}

On peut aussi former la version non standard des ensembles usuels. On construit
ainsi
\begin{enumerate}
	\item $\nonStandard{\naturel}$: les naturels non standards, à partir de
		$\naturel$.
	\item $\nonStandard{\integer}$: les entiers non standards, à partir de
		$\integer$.
	\item $\nonStandard{\rational}$: les rationnels non standards, à partir de
		$\rational$.
\end{enumerate}

Les naturels non standards sont par exemples les classes d'équivalences de
suites
$\GSsequence{N}{n}{\naturel}$ où $\GSsetDef{n \in \naturel}{N_{n} \in \naturel}
\in \mathcal{U}$, c'est-à-dire les classes d'équivalences des suites dont
presque tous les éléments appartiennent à $\naturel$.

Par exemple, $[(\frac{1}{2}, \frac{1}{4}, 1, 2, 3, \dots)]_{\mathcal{U}}$ est un
naturel non standard.

Nous allons continuer avec les ensembles hyperfinis.

\begin{definition} [Ensemble hyperfini]
	Soit un ensemble interne $\spanspace{A_{n}} \subseteq \hyperreal$. On dit
	que $\spanspace{A_{n}}$ est \textbf{un ensemble hyperfini} si
	\begin{equation}
		\GSsetDef{n \in \naturel}{A_{n} \text{ est fini}} \in \mathcal{U}
	\end{equation}

	La cardinalité d'un ensemble hyperfini $\spanspace{A_{n}}$ est le naturel
	non standard $\spanspace{\cardinal{A_{n}}}$ où $\cardinal{A_{n}}$ est la
	cardinalité de $A_{n}$.
\end{definition}

\begin{remarque}
	La cardinalité d'un ensemble hyperfini $\spanspace{A_{n}}$ est bien un
	naturel non standard car l'ensemble des naturels décrivant les $A_{n}$ est
	dans l'ultrafiltre, d'où la classe d'équivalence de la suite des cardinaux
	est bien dans l'ultrafiltre, c'est-à-dire dans les naturels non standards.
\end{remarque}

\begin{exemple}
	Soit $N \in \naturel$ et construisons l'élément $[N] \in \naturel$ à partir
	de la suite constante de naturels de constante $N$, ie $(N, N, N, N, N,
	\cdots)$.

	Posons $A_{n} = \GSset{0, \frac{1}{N}, \frac{2}{N}, \cdots, \frac{N -
	1}{N}, 1} \subseteq \naturel$ et construisons l'ensemble interne
	$\spanspace{A_{n}}$.

	Alors, pour tous $n \in \naturel$, $A_{n}$ est fini, de cardinalité $N + 1$.
	Donc $\spanspace{A_{n}}$ est hyperfini de cardinalité $[N + 1]$.
\end{exemple}

Remarquons que les ensembles hyperfinis héritent de différentes propriétés des
ensembles finis. Par exemple
\begin{proposition}
	Soient $\spanspace{A_{n}}$ et $\spanspace{B_{n}}$ un ensemble hyperfini. Alors
	\begin{enumerate}
		\item $\spanspace{A_{n}}$ et $\spanspace{B_{n}}$ possèdent un plus petit élément.
		\item $\spanspace{A_{n}}$ et $\spanspace{B_{n}}$ possèdent un plus grand élément.
		\item $\spanspace{A_{n}} \union \spanspace{B_{n}}$ est hyperfini.
	\end{enumerate}
\end{proposition}

\begin{proof}
	Laissée au lecteur.
\end{proof}

D'autres notions peuvent être étendu aux ensembles hyperfinis à partir des
ensembles finis.

Prenons un ensemble hyperfini $A = \spanspace{A_{n}}$ et une fonction interne
$f = \spanspace{f_{n}}$. On peut définir la somme des images de $f$ sur $A$
comme
\footnote{Cela a bien du sens que $A_{n} \subseteq \real$ et
$f_{n} : \real \rightarrow \real$}
\begin{equation}
	\sum_{a \in A} f(a) = \spanspace{\sum_{a_{n} \in A_{n}} f_{n}(a_{n})}
\end{equation}


% Besoin de définir bijections internes et de cardinalité interne.

%Suites aux propriétés ci-dessus, on est amené à une définition équivalente
%d'ensemble hyperfini.

%\begin{proposition}
	%Soit $\spanspace{A_{n}}$ un ensemble interne. Les assertions suivantes sont
	%équiavlentes:
	%\begin{enumerate}
		%\item $\spanspace{A_{n}}$ est un ensemble hyperfini de cardinalité $[N]$
	%\end{enumerate}
%\end{proposition}

\section{Calcul infinitésimal}

Maintenant que nous avons défini la notion d'infinitésimaux, il est intéressant
de faire l'analogie avec le calcul infinitésimal. Nous allons voir que la notion
d'infinitésimaux n'est pas anodine. En fait, les hyperréels ont été développés
pour donner une définition précise d'\textit{infinitésimaux}.

\begin{proposition}
	Soit $f : \real \rightarrow \real$ et $a \in \real$. Les assertions
	suivantes sont équivalentes:
	\begin{enumerate}
		\item $f$ est continue en $a$
		\item $\nonStandard{f}(x) \approx \nonStandard{f}(i(a))$ pour tous $x
			\in \hyperreal$ tel que $x \approx i(a)$.
	\end{enumerate}
\end{proposition}

\ifdefined\outputproof
\begin{proof}

\end{proof}
\fi

La continuité de la composition de deux fonctions continues devient alors
évidente.

\begin{proposition}
	Soient $f, g : \real \rightarrow \real$. Soit $a \in \real$ tel que $g$ est
	continue en $a$ et $f$ est continue en $g(a)$.

	Alors $f \circ g$ est continue en $a$.
\end{proposition}

\ifdefined\outputproof
\begin{proof}
	Si $x \approx a$, alors $\nonStandard{g}(x) \approx \nonStandard{g}(a)$ ($g$
	continue en $a$) et
	cela induit que $\nonStandard{f}(\nonStandard{g}(x))
	\approx \nonStandard{f}(\nonStandard{g}(a))$ car $f$ est continue en $g(a)$.

	On a donc bien montré que $f \circ g$ est continue en $a$ car pour tous $x \approx
	a$, $(\nonStandard{f} \circ \nonStandard{g}) (x) \approx (\nonStandard{f}
	\circ \nonStandard{g})(a)$.
\end{proof}
\fi

La propriété qu'une fonction continue sur un compact atteint ses bornes est
aussi évident:

\begin{proposition}
	Soit $f : K \rightarrow \real$ une fonction continue et $K$ un compact de
	$\real$.

	Alors $f$ atteint ses bornes.
\end{proposition}

\ifdefined\outputproof
\begin{proof}

\end{proof}
\fi

On peut aussi donner une équivalence pour la notion de continuité uniforme.

\begin{proposition}
	Soit $f : \real \rightarrow \real$ et $A \subseteq \real$. Les assertions
	suivantes sont équivalentes:

	\begin{enumerate}
		\item $f$ est uniformément continue sur $A$.
		\item $\nonStandard{f}(x) \approx \nonStandard{f}(y)$ pour tous éléments
			non standards $x, y \in \nonStandard{A}$.
	\end{enumerate}
\end{proposition}

\ifdefined\outputproof
\begin{proof}

\end{proof}
\fi

On obtient alors une preuve simple de la proposition suivante:

\begin{proposition}
	Soit $f : K \rightarrow \real$ une fonction continue et $K$ un interval compact.
	Alors $f$ est uniformément continue sur $K$.
\end{proposition}

\ifdefined\outputproof
\begin{proof}

\end{proof}
\fi

La définition de dérivée a également une équivalence hyperréelles.

\begin{proposition}
	Soit $f : \real \rightarrow \real$ et $a \in \real$. Les assertions
	suivantes sont équivalentes:

	\begin{enumerate}
		\item $f$ est dérivable en $a$
		\item il existe $b \in \real$ tel que pour tous $x \in \hyperreal$
			\begin{equation}
				x \approx [a] \implies \frac{\nonStandard{f}(x) -
				\nonStandard{f}([a])}{x - [a]} \approx b
			\end{equation}
			Si ce $b$ existe, alors, $b = f'(a)$.
	\end{enumerate}
\end{proposition}

\ifdefined\outputproof
\begin{proof}

\end{proof}
\fi

% Chain Rule et solution équation différentielle.

\chapter{Superstructures et mesures de Loeb}

Rappelons que la construction des hyperréels a comme base un ultrafiltre et une
relation d'équivalence. On prend alors le quotient pour obtenir $\hyperreal$.

On peut remarquer que cette construction est valable pour un ensemble quelconque
$S$.

En effet, soit $S$ un ensemble quelconque. Soit $\mathcal{U}$ un ultrafiltre non
principal sur $\naturel$. On construit alors les hyperéléments de $S$, dont
l'ensemble est noté $\hyperstructure{S}$ construit comme
$S/\mathcal{U}$.

Comme les notions d'ensembles internes, de fonctions internes, d'ensembles
hyperfinis, d'éléments non standards (etc) ont été définies en utilisant que des
notions ensembles, on peut élargir ces notions à l'hyperstructure
$\hyperstructure{S}$.

Pour la suite, nous fixons un ensemble $S$.

%Cependant, dès qu'on a définies les fonctions internes $\spanspace{f} :
%\hyperstructure{S} \rightarrow \hyperstructure{S}$, on s'amuser à

\begin{definition}
	Soit un ensemble $S$. On définit récursivement la suite $(V_{n}(S))_{n
	\naturel}$ tel que
	\begin{enumerate}
		\item $V_{0}(S) = S$
		\item $V_{n + 1}(S) = V_{n}(S) \union \powerSet{V_{n}(S)}$
	\end{enumerate}

	On définit alors \textbf{la superstructure de $S$}, noté $V(S)$ comme $V(S)
	= \union_{n \in \naturel}V_{n}(S)$.

	Le rang d'un élément $x \in V(S)$ est le plus petit $n \in \naturel$ tel que
	$x \in V_{n}(S)$.
\end{definition}

La motivation des superstructures est de pouvoir parler d'éléments de l'ensemble
$S$, de fonctions $f : S \rightarrow S$ (élément de $V_{3}(S)$) et de fonctions
$f' : \GSset{f : S \rightarrow S} \rightarrow \GSset{f : S \rightarrow S}$
(élément de $V_{6}(S)$) comme des éléments d'un seul et même ensemble.

Le rang d'un élément permet de \textit{reclasser} l'élément dans sa catégorie
'usuelle'. Par exemple, une fonction de $S$ dans $S$ est de rang $3$, un couple
$(x, y) = \GSset{x, \GSset{x, y} } \in S^{2}$ est de rang $1$, un élément de $S$
est de rang $0$.

$\hyperstructure{S}$ étant aussi un ensemble, il est possible de définir
$V(\hyperstructure{S})$.

Nous pouvons voir les différentes constructions grâce au diagramme:

\begin{displaymath}
	\xymatrix{ 	S \ar[d]^{1} \ar[r]^{2} & \hyperstructure{S} \ar@{.>}[d]^{3} \\
	V(S) \ar@{~>}[r]^{4} & V(\hyperstructure{S})}
\end{displaymath}

où $1$ est la construction de la superstructure de $S$ et $2$ celle
l'hyperstructure de $S$.
$3$ représente la construction de la superstructure de $\hyperstructure{S}$. La
dernière flèche, c'est-à-dire la $4$, est l'étape que nous allons réaliser:
construire certains éléments de $V(\hyperstructure{S})$ à partir d'éléments de
$V(S)$, la superstructure de $S$.

Nous allons tenter de définir certains éléments de $V(\hyperstructure{S})$ et
certains d'entres-eux seront appelés \textit{élément internes}.

Nos éléments internes 'généralisés' vont être des éléments de
$V(\hyperstructure{S})$ construits à partir de suites
$\GSsequence{A}{n}{\naturel}$ d'éléments de $V(S)$. Remarquons que les $A_{n}$
sont de natures quelconques, c'est-à-dire que $A_{0}$ peut être un élément de
$S$, $A_{1}$ un couple, $A_{2}$, une fonction, etc.


\begin{definition} [Suite bornée]
	Soit $\GSsequence{A}{n}{\naturel}$ une suite d'éléments de $V(S)$.

	On dit que la suite $\GSsequence{A}{n}{\naturel}$ est \textbf{bornée} par $p
	\in \naturel$ si
	\begin{equation}
		\naturel = \union_{i \in \GSset{0, \cdots, p}} \GSsetDef{n
		\in \naturel}{A_{n} \text{ est de rang $i$}}
	\end{equation}

	Posons $B_{i} = \GSsetDef{n \in \naturel}{A_{n} \text{ est de rang $i$}}$

	Comme $\naturel \in \mathcal{U}$ et que l'union est finie, il est sûr qu'un
	des $B_{i}$ est dans $\mathcal{U}$ et celui-ci est unique.

	Ce naturel $i$ est appelé \textbf{le rang de la suite
		$\GSsequence{A}{n}{\naturel}$}
\end{definition}

Une suite de $V(S)$ est donc bornée si le rang de ses éléments n'explose pas.

Remarquons que dans le cas des hyperréels, nous n'avons travaillé qu'avec des
éléments de rang $1$ pour les ensembles internes et $3$ pour les fonctions
internes.

A partir de suites bornées, nous allons maintenant définir des éléments de
$V(\hyperstructure{S})$.

Soit $\GSsequence{A}{n}{\naturel}$ une suite bornée d'éléments de $V(S)$.

Si $\GSsequence{A}{n}{\naturel}$ est de rang $0$, chaque élément de la suite est
simplement un élément $s \in S$. D'où, l'hyperélément $[A_{n}] \in
\hyperstructure{S}$.

Supposons que nous avons défini les éléments $\spanspace{B_{n}}$ pour les
éléments de $V(S)$ de rang $i$ et définissons les éléments $\spanspace{B_{n}}$
pour les éléments de $V(S)$ de rang $i + 1$.

Si $\GSsequence{A}{n}{\naturel}$ est de rang $i + 1$, on définit l'hyperélément
$\spanspace{A_{n}}$ comme

\begin{equation}
	\GSsetDef{\spanspace{B_{n}}}{\GSsequence{B}{n}{\naturel} \text{ est de rang $\leq i$}
		\wedge \GSsetDef{n \in \naturel}{B_{n} \in A_{n}} \in \mathcal{U}}
\end{equation}

Les éléments $\spanspace{A_{n}}$ définis ainsi, par récurrence sur le rang, sont
dans $V(\hyperstructure{S})$. Nous allons nous servir de ces éléments pour
définir nos éléments internes.

\begin{definition}
	Soit $x \in V(\hyperstructure{S})$.

	On dit que $x$ est \textbf{interne} s'il est de la forme $\spanspace{A_{n}}$
	résultant de la construction précédente.

	Si $\GSsetDef{n \in \naturel}{A_{n} \text{ est fini}} \in \mathcal{U}$, $x$
	est \textbf{hyperfini}.

	Un ensemble interne $\nonStandard{x}$ de la forme $\spanspace{A}$ est appelé
	\textbf{standard}.
\end{definition}

\chapter{Saturation et topologie}

Les hyperstructures et superstructures de $S$ ont été définies à partir de
familles basés sur $\naturel$ et un ultrafiltre non principal $\mathcal{U}$ sur
$\naturel$.

On peut alors définir les mêmes hyperstructures et superstructures à partir
d'une famille basée sur un ensemble infini quelconque $I$ et un ultrafiltre
$\mathcal{U}$ sur $I$.

Fixons un ensemble $S$ quelconque, $I$ un ensemble infini quelconque et
$\mathcal{U}$ un ultrafiltre non principal sur $I$. Construisons
$\hyperstructure{S}$ et $V(\hyperstructure{S})$ depuis $S$, $I$ et
$\mathcal{U}$.

\begin{definition}
	Soit $\kappa$ un cardinal infini.
	On dit que $V(\hyperstructure{S})$ est \textbf{$\kappa$-saturé} si pour tout
	ensemble $\Gamma$ tel que $\cardinal{\Gamma} < \kappa$, toute famille
	$\GSsequence{A}{\gamma}{\Gamma}$ est d'intersection non vide.

	On dit ausi que $V(\hyperstructure{S})$ est \textbf{un modèle
	$\kappa$-saturé.}
\end{definition}

\section{Topologie générale}

Soit $(X, \tau)$ un espace topologique, et posons $S = \real \union X$.

Construisons alors un modèle $\kappa$-saturé $V(\hyperstructure{S})$.

\begin{definition}
	Soit $a \in X$. On définit \textbf{le monade de $x$} comme le sous-ensemble
	$\mu(a)$ de $\hyperstructure{X}$ défini par:
	\begin{equation}
		\mu(a) = \inter \GSsetDef{\hyperstructure{\mathcal{O}} \subseteq
		\hyperstructure{X}}{a \in \mathcal{O} \wedge \mathcal{O} \in \tau}
	\end{equation}
\end{definition}

\begin{definition}
	Soit $x \in X$. On dit que \textbf{$x$ est proche d'un standard} s'il existe
	$a \in X$ tel que $x \in \mu(a)$, c'est-à-dire qu'il existe $a \in A$ tel
	que $x$ est dans le monade de $a$. On dit aussi que \textbf{$x$ est proche du
	standard $a$}.

	L'ensemble de tous les points de proches d'un standard est noté
	$ns(\hyperstructure{X})$.
\end{definition}

On en vient maintenant à la caractérisation des ouverts, fermés et compacts de
de $X$.

\begin{proposition}
	Soit $A$ un sous-ensemble de $X$. Alors
	\begin{enumerate}
		\item $A$ est ouvert dans $(X, \tau)$ ssi pour tout $a \in A$, $\mu(a)
			\subseteq \hyperstructure{A}$.
		\item $A$ est fermé dans $(X, \tau)$ ssi quelque soit $x \in
			\hyperstructure{A}$ proche d'un standard $a \in X$, $a \in A$.
		\item $A$ est compact ssi tout $x \in \hyperstructure{A}$ est proche
			d'un standard $a \in A$.
	\end{enumerate}
\end{proposition}

\ifdefined\outputproof
\begin{proof}
	\begin{enumerate}
		\item $(1)$ Si $A$ est ouvert dans $(X, tau)$, alors $\mu(a) \subseteq
			\hyperstructure{A}$ par définition du monade de $a$.

			Supposons maintenant que $A$ n'est pas ouvert et montrons qu'il
			existe $a \in A$ tel qe $\mu(a) \inter \hyperstructure{A} \neq
			\emptyset$. Comme $A$ n'est pas ouvert, il existe $a \in A$ tel que
			pour tout ouvert $\mathcal{O}$ contenant $a$, $\mathcal{O} \inter
			\comp{A} \neq \emptyset$.

			Par saturation, en indiçant par les ouverts de $(X, tau)$ la suite
			d'ensemble
			\begin{equation}
				A_{\mathcal{O}} := \GSsetDef{\hyperstructure{\mathcal{O}} \inter
				\hyperstructure{(\comp{A})}}{a \in \mathcal{O} \wedge
				\mathcal{O} \in \tau}
			\end{equation}
			on obtient
			\begin{equation}
				\inter_{\mathcal{O} \in \tau} A_{\mathcal{O}} \neq \emptyset
			\end{equation}
			Il est alors clair que tout élément $x \in \inter_{\mathcal{O} \in
		\tau} A_{\mathcal{O}}$ n'appartient pas à $A$ mais est bien dans $\mu(a)$.

		\item $(2)$ Soit $A$ compact et supposons que pour tout $a \in A$, il
			existe $x \in X$ et un voisinage ouvert $V_{a}$ contenant $a$ tel
			que $x \notin \hyperstructure{V_{a}}$.

			La fmaille $\GSsequence{V}{a}{A}$ étant un recouvrement de $A$ et
			$A$ étant compact, il existe un sous-recouvrement fini $V_{a_{1}},
			\cdots, V_{a_{n}}$ de $A$ ie $A \subseteq V_{a_{1}} \union \cdots
			\union V_{a_{n}}$. Comme le passage à la version non standard est
			stable par union finie, on a $\hyperstructure{A} \subseteq
			\hyperstructure{V_{a_{1}}} \union \cdots \union
			\hyperstructure{V_{a_{n}}}$. Contradiction avec le fait que $a$
			appartient à aucun des $\hyperstructure{V_{a_{i}}}$.
			\ldots
	\end{enumerate}
\end{proof}
\fi

Rappelons qu'un espace topologique $(X, \tau)$ est \textbf{Hausdorff} s'il est
séparable, c'est-à-dire que tout point $x, y \in X$, il existe des voisinages
ouverts $V_{x}$ et $V_{y}$ disjoints.

\begin{proposition}
	Soit $(X, \tau)$ un espace topologique de Hausdorff. Alors, tout point $x
	\in ns(\hyperstructure{X})$ est proche d'un unique non standard $a$.
\end{proposition}

\ifdefined\outputproof
\begin{proof}
	Soit $x \in ns(\hyperstructure{X})$ ie il existe $a \in X$ tel que $x \in
	\mu(a)$.

	S'il existait $b \in X$ tel que $x \in \mu(b)$, alors $a$ et $b$
	appartiennent aux mêmes ouverts. Or, il existe des ouverts les séparant.
\end{proof}
\fi

Vu l'unicité, on appelle cet unique $a$ la \textbf{partie
standard} de $x$, et est notée $\standardPart{x}$ ou $\standardPartExpo{x}$.

On peut ainsi définir la fonction
\begin{equation}
	st : ns(\hyperstructure{X}) \rightarrow X : x \rightarrow st(x)
\end{equation}

Cette notion étend bien la notion de non standard définie pour les hyperréels.

\begin{definition}
	Soient $x, y \in X$. On dit que $x$ et $y$ sont \textbf{infiniment proches}
	s'ils sont proches du même standard $a \in X$.
\end{definition}

On obtient alors une autre formulation des équivalences données ci-dessous pour
les ouverts, fermés et compacts.

\begin{enumerate}
	\item $A$ est ouvert ssi $st^{-1}(A) \subseteq A$
	\item $A$ est fermé ssi $\hyperstructure{A} \inter ns(\hyperstructure{X})
		\subseteq st^{-1}(A)$.
	\item $A$ est compact ssi $\hyperstructure{A} \subseteq st^{-1}(A)$.
\end{enumerate}

\begin{proposition}
	Soit $(X, \tau)$ un espace de Hausdorff et soit $A$ un sous ensemble interne
	de $\hyperstructure{X}$.

	Alors $st(A)$ est fermé dans $(X, \tau)$.
\end{proposition}

\ifdefined\outputproof
\begin{proof}
	Soit $a \in \adh{A}$. On a, pour tout voisinage de $a$ ouvert
	$\mathcal{O}_{a}$ , $\mathcal{O}_{a} \inter st(A) \neq \emptyset$.
	D'où cette propriété reste vraie par intersection finie. Par saturation, la
	famille $\GSsequence{B}{a}{A}$ définie par
	\begin{equation}
		B_{a} = \GSsetDef{\hyperstructure{\mathcal{O}_{a}} \inter
	A}{\mathcal{O}_{a} \in \tau}
	\end{equation}
	est d'intersection non vide.

	Soit $x$ un élément de l'intersection, en particulier. Alors, $x \in \mu(a)$ car $x \in
	\mathcal{O}_{a}$ pour tout $\mathcal{O}_{a} \in \tau$. D'où, $st(x) = a$ et
	$a \in st(A)$ car $x \in A$ en particulier.
\end{proof}
\fi

\begin{proposition}
	Soit $(X, \tau_{X})$ et $(Y, \tau_{Y})$ deux espaces topologiques. Soit $f :
	X \rightarrow Y$ et soit $a \in X$.
	Les assertions suivantes sont équivalentes:
	\begin{enumerate}
		\item $f$ est continue en $a$.
		\item $\hyperstructure{f}(\mu(a)) \subseteq \mu(f(a))$
	\end{enumerate}
\end{proposition}

\ifdefined\outputproof
\begin{proof}
	$(1) \implies (2)$:

	Soit $x \in \mu(a)$. Montrons que $\hyperstructure{f}(x) \in \mu(f(a))$,
	c'est-à-dire que $\hyperstructure{f}(x) \in
	\hyperstructure{\mathcal{O}_{f(a)}}$ pour tout ouvert
	$\mathcal{O}_{f(a)}$ contenant $f(a)$.
\end{proof}
\fi

\appendix

\chapter{Mesures}

Nous supposons que nous travaillons dans un espace mesurable $(X, \mathcal{A})$.

%\begin{definition}
	%Soit $(X, \mathcal{A})$ un espace mesurable.
	%Soit $\mu : \mathcal{A} \rightarrow [-\infty, + \infty]$ une fonction.

	%On dit que $\mu$ est \textbf{additive} ou \textbf{finie additive} si pour
	%tout $A, B \in \mathcal{A}$,
	%\begin{equation}
		%\mu(A \union B) = \mu(A) + \mu(B)
	%\end{equation}
%\end{definition}

\begin{definition}
	Soit $(X, \mathcal{A})$ un espace mesurable.
	Soit $\mu : \mathcal{A} \rightarrow [0, + \infty]$ une fonction.

	On dit que $\mu$ est \textbf{mesure} si
	\begin{enumerate}
		\item $\mu(\emptyset) = 0$
		\item pour toute famille $\GSsequence{A}{n}{\naturel}$ d'éléments $2$ à
			$2$ disjoints de $\mathcal{A}$
			\begin{equation}
				\mu(\union_{n \in \naturel} A_{n}) = \sum_{n \in \naturel}
				\mu(A_{n})
			\end{equation}
	\end{enumerate}
\end{definition}

\begin{definition}
	Soit $(X, \mathcal{A})$ un espace mesurable.
	Soit $\mu : \mathcal{A} \rightarrow [0, + \infty]$ une fonction.

	On dit que $\mu$ est \textbf{une mesure (additive) finie} si $\mu$ est une
	mesure et si $\mu(X)$ est fini.
\end{definition}

Notre construction des hyperréels sera basée sur une mesure finie additive
respectant une propriété supplémentaire.

Donnons d'abord une première proposition.

\begin{proposition}
	Il existe des mesures additives finies $m : \powerSet{\naturel} \rightarrow
	\GSset{0, 1}$ tel que
	\begin{enumerate}
		\item $m$ est nul sur tout sous-ensemble fini de $\naturel$
		\item $m(\naturel) = 1$
	\end{enumerate}
\end{proposition}

\ifdefined\outputproof
\begin{proof}

\end{proof}
\fi

Remarquons que pour une telle mesure $m$, nous n'avons pas $m(A) = 1$ et
$m(\comp{A}) = 1$. En effet, si c'était le cas, on aurait $m(\naturel) = 2$. Or
$m(\naturel) = 1$.

\chapter{Construction des hyperréels et des superstructures grace à une mesure}

Pour le reste du document, nous fixons une mesure additive finie $m$ répondant
aux critères précédent.

\begin{notation}
	Soit $\real$ l'ensemble des nombres réels. On note $\mathcal{S}$ l'ensemble
	des suites réelles.
\end{notation}

\begin{definition}
	On définit la relation $\approx$ sur l'ensemble $\mathcal{S}$ par
	\begin{equation}
		(a_{n}) \approx (b_{n}) \equiv m(\GSsetDef{n \in \naturel}{a_{n} = b_{n}})
	\end{equation}
\end{definition}

\begin{proposition}
	La relation $\approx$ définie précédemment est une relation d'équivalence sur
	$\mathcal{S}$.
\end{proposition}

\ifdefined\outputproof
\begin{proof}

\end{proof}
\fi

\begin{definition}
	Soit $\mathcal{S}$ l'ensemble des suites réelles et $\approx$ la relation
	d'équivalence définie précédemment.

	On appelle l'ensemble quotient $\mathcal{S} / \approx$ \textbf{l'ensemble des hyperréels} ou
	\textbf{l'ensemble des réels non standards}.
\end{definition}

Construisons maintenant des opérations sur $\hyperreal$.

\begin{definition}
	Soit $\hyperreal$ l'ensemble des hyperréels et soient $[(a_{n})_{n \in
	\naturel}]$ et $[(b_{n})_{n \in \naturel}]$ deux hyperréels.

	On définit l'opération $+_{\hyperreal}$ tel que
	\begin{equation}
		[(a_{n})_{n \in \naturel}] +_{\hyperreal} [(b_{n})_{n \in \naturel}] = [(a_{n} +
		b_{n})_{n \in \naturel}]
	\end{equation}
	et l'opération $._{\hyperreal}$ tel que
	\begin{equation}
		[(a_{n})_{n \in \naturel}] ._{\hyperreal}  [(b_{n})_{n \in \naturel}] =
		[(a_{n} .
		b_{n})_{n \in \naturel}]
	\end{equation}

	Par abus de notations, on omettra les indices sur les opérations.
\end{definition}

\begin{definition}
	Soit $\hyperreal$ l'ensemble des hyperréels et soient $[(a_{n})_{n \in
	\naturel}]$ et $[(b_{n})_{n \in \naturel}]$ deux hyperréels.

	On définit la relation d'ordre $<_{\hyperreal}$ tel que
	\begin{equation}
		[(a_{n})_{n \in \naturel}] <_{\hyperreal} [(b_{n})_{n \in \naturel}]
	\end{equation}
	ssi
	\begin{equation}
		m(\GSsetDef{n \in \naturel}{a_{n} < b_{n}}) = 1
	\end{equation}
\end{definition}

Remarquons que ces deux opérations et l'ordre défini permettent de retrouver des
caractéristiques de $\real$.

\begin{proposition}
	Soit $\hyperreal$ l'ensemble des hyperréels, $+$ et $.$ les deux opérations
	d'addition et de multiplication définies précédemment.
	Soient $0_{\hyperreal} = [0]$, $1_{\hyperreal} = [1]$ et $<_{\hyperreal}$
	l'ordre défini précédemment sur $\hyperreal$.

	Alors $(\hyperreal, +, ., 0_{\hyperreal}, <_{\hyperreal})$ forme un corps
	ordonné.
\end{proposition}

\ifdefined\outputproof
\begin{proof}

\end{proof}
\fi

\end{document}
