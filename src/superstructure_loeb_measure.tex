\chapter{Superstructures et mesures de Loeb}

Rappelons que la construction des hyperréels a comme base un ultrafiltre et une
relation d'équivalence. On prend alors le quotient pour obtenir $\hyperreal$.

On peut remarquer que cette construction est valable pour un ensemble quelconque
$S$.

En effet, soit $S$ un ensemble quelconque. Soit $\mathcal{U}$ un ultrafiltre non
principal sur $\naturel$. On construit alors les hyperéléments de $S$, dont
l'ensemble est noté $\hyperstructure{S}$ construit comme
$S/\mathcal{U}$.

Comme les notions d'ensembles internes, de fonctions internes, d'ensembles
hyperfinis, d'éléments non standards (etc) ont été définies en utilisant que des
notions ensembles, on peut élargir ces notions à l'hyperstructure
$\hyperstructure{S}$.

Pour la suite, nous fixons un ensemble $S$.

%Cependant, dès qu'on a définies les fonctions internes $\spanspace{f} :
%\hyperstructure{S} \rightarrow \hyperstructure{S}$, on s'amuser à

\begin{definition} [Superstructure]
	Soit un ensemble $S$. On définit récursivement la suite $(V_{n}(S))_{n
	\in \naturel}$ tel que
	\begin{enumerate}
		\item $V_{0}(S) = S$
		\item $V_{n + 1}(S) = V_{n}(S) \union \powerSet{V_{n}(S)}$
	\end{enumerate}

	On définit alors \textbf{la superstructure de $S$}, notée
	$\superstructure{S}$ comme

	\begin{equation}
		\superstructure{S} = \union_{n \in \naturel}V_{n}(S)
	\end{equation}

	Le rang d'un élément $x \in \superstructure{S}$ est le plus petit $n \in
	\naturel$ tel que $x \in V_{n}(S)$.
\end{definition}

La motivation des superstructures est de pouvoir parler d'éléments de l'ensemble
$S$, de fonctions $f : S \rightarrow S$ (élément de $V_{3}(S)$) et de fonctions
$f' : \GSset{f : S \rightarrow S} \rightarrow \GSset{f : S \rightarrow S}$
(élément de $V_{6}(S)$) comme des éléments d'un seul et même ensemble.

Le rang d'un élément permet de \textit{reclasser} l'élément dans sa catégorie
\og usuelle \fg. Par exemple, une fonction de $S$ dans $S$ est de rang $3$, un couple
$(x, y) = \GSset{x, \GSset{x, y} } \in S^{2}$ est de rang $1$, un élément de $S$
est de rang $0$.

$\hyperstructure{S}$ étant aussi un ensemble, il est possible de définir
$\superstructure{\hyperstructure{S}}$.

Nous pouvons voir les différentes constructions grâce au diagramme:

\begin{displaymath}
	\xymatrix{ S \ar[d]^{1} \ar[r]^{2} & \hyperstructure{S} \ar@{.>}[d]^{3} \\
	\superstructure{S} \ar@{~>}[r]^{4} & \superstructure{\hyperstructure{S}}}
\end{displaymath}

où $1$ est la construction de la superstructure de $S$ et $2$ celle
l'hyperstructure de $S$.
$3$ représente la construction de la superstructure de $\hyperstructure{S}$. La
dernière flèche, c'est-à-dire la $4$, est l'étape que nous allons réaliser:
construire certains éléments de $\superstructure{\hyperstructure{S}}$ à partir
d'éléments de $\superstructure{S}$, la superstructure de $S$. Ces éléments
seront appelés \textit{élément internes}.

Nos éléments internes \og généralisés \fg vont être des éléments de
$\superstructure{\hyperstructure{S}}$ construits à partir de suites
$\GSsequence{A}{n}{\naturel}$ d'éléments de $\superstructure{S}$. Remarquons que
les $A_{n}$ sont de natures quelconques, c'est-à-dire que $A_{0}$ peut être un
élément de $S$, $A_{1}$ un couple, $A_{2}$, une fonction, etc.


\begin{definition} [Suite bornée]
	Soit $\GSsequence{A}{n}{\naturel}$ une suite d'éléments de $\superstructure{S}$.

	On dit que la suite $\GSsequence{A}{n}{\naturel}$ est \textbf{bornée} par $p
	\in \naturel$ si
	\begin{equation}
		\naturel = \union_{i \in \GSset{0, \cdots, p}} \GSsetDef{n
		\in \naturel}{A_{n} \text{ est de rang $i$}}
	\end{equation}

	Posons $B_{i} = \GSsetDef{n \in \naturel}{A_{n} \text{ est de rang $i$}}$.

	Comme $\naturel \in \mathcal{U}$ et que l'union est finie, il est sûr qu'un
	des $B_{i}$ est dans $\mathcal{U}$ et celui-ci est unique.

	Ce naturel $i$ est appelé \textbf{le rang de la suite
		$\GSsequence{A}{n}{\naturel}$}.
\end{definition}

Une suite de $\superstructure{S}$ est donc bornée si le rang de ses éléments
n'explose pas.

Remarquons que dans le cas des hyperréels, nous n'avons travaillé qu'avec des
éléments de rang $1$ pour les ensembles internes et $3$ pour les fonctions
internes.

A partir de suites bornées, nous allons maintenant définir des éléments de
$\superstructure{\hyperstructure{S}}$.

Soit $\GSsequence{A}{n}{\naturel}$ une suite bornée d'éléments de
$\superstructure{S}$.

Si $\GSsequence{A}{n}{\naturel}$ est de rang $0$, chaque élément de la suite est
simplement un élément $s \in S$. D'où, l'hyperélément $\hyperelementSequence{A}
\in \hyperstructure{S}$.

Supposons que nous avons défini les éléments $\hyperelementSequence{B}$ pour les
éléments de $\superstructure{S}$ de rang $i$ et définissons les éléments
$\hyperelementSequence{B}$ pour les éléments de $\superstructure{S}$ de rang $i
+ 1$.

Si $\GSsequence{A}{n}{\naturel}$ est de rang $i + 1$, on définit l'hyperélément
$\hyperelementSequence{A}$ comme

\begin{equation}
	\GSsetDef{\hyperelementSequence{B}}{\GSsequence{B}{n}{\naturel} \text{ est de rang $\leq i$}
		\wedge \GSsetDef{n \in \naturel}{B_{n} \in A_{n}} \in \mathcal{U}}
\end{equation}

Les éléments $\hyperelementSequence{A}$ définis ainsi, par récurrence sur le rang, sont
dans $\superstructure{\hyperstructure{S}}$. Nous allons nous servir de ces éléments pour
définir nos éléments internes.

\begin{definition}
	Soit $x \in \superstructure{\hyperstructure{S}}$.

	On dit que $x$ est \textbf{interne} s'il est de la forme $\spanspace{A_{n}}$
	résultant de la construction précédente.

	Si $\GSsetDef{n \in \naturel}{A_{n} \text{ est fini}} \in \mathcal{U}$, $x$
	est \textbf{hyperfini}.

	Un ensemble interne $\nonStandardVersion{x}$ de la forme $\spanspace{A}$ est appelé
	\textbf{standard}.
\end{definition}
