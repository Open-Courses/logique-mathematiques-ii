\chapter{Analyse non standard et hyperréels}

\label{chapter_hyperreal}

\section{Construction}

Nous allons construire l'ensemble des hyperréels qui étendra le corps des réels.
Nous définirons ensuite les ensembles internes, les éléments infinitésimaux et
les éléments finis. Nous pourrons donner une définition précise de la notion
d'infénitésimal à travers les éléments infinitésimaux et nous verrons le lien
avec les suites.

On pourra remarquer que la notion d'hyperréel peut être étendue à n'importe quel
ensemble $S$. On parlera alors de l'hyperstructure de $S$.

Soit $\real$ l'ensemble des réels. Soit $\mathcal{U}$ un ultrafiltre non
principal sur $\naturel$.

On définit la relation d'équivalence $\sim$ sur $\real^{\naturel}$ tel que
\begin{equation}
	\GSsequence{x}{n}{\naturel} \sim \GSsequence{y}{n}{\naturel} \equivalence \GSsetDef{n \in \naturel}{x_{n} = y_{n}} \in \mathcal{U}
\end{equation}

On obtient alors \textbf{l'ultrapuissance de $\real$} noté $\real^{\mathcal{U}}$,
ou encore $\hyperreal$. Cet ultrapuissance est appelé \textbf{l'ensemble des
hyperréels}.

On note les éléments de $\hyperreal$ par $\hyperelement{x}$,
$\hyperelementSequence{x}$ quand nous devons spécifier de quelle suite
l'hyperréel a été construit. Nous utiliserons plus rarement la notation
$\hyperelementUltra{x}{\mathcal{U}}$ pour $\hyperelement{x}$ (resp.
$\hyperelementSequenceUltra{x}{\mathcal{U}}$) quand nous voulons spécifier l'ultrafiltre
sous-jacent.

Nous ferons la distinction quand cela est nécessaire mais la plupart du temps,
nous utiliserons $\hyperelement{x}$.

\begin{definition} [Hyperréel de même forme]
	Soient $\hyperelement{x}$, $\hyperelement{y}$ deux hyperréels. On dit que
	$\hyperelement{x}$ et $\hyperelement{y}$ sont \textbf{de la même forme}
	s'ils appartiennent à la même classe d'équivalence. On dit aussi que
	\textbf{$\hyperelement{x}$ (resp. $\hyperelement{y}$) est de la forme
	$\hyperelement{y}$ (resp. $\hyperelement{x}$)}.
\end{definition}

On peut étendre les opérations et l'ordre usuels de $\real$ à $\hyperreal$ de la façon
suivante:

Soient $\hyperelementSequenceFull{x}$ et $\hyperelementSequenceFull{y}$
\begin{enumerate}
	\item $\hyperelementSequenceFull{x} +_{\hyperreal} \hyperelementSequenceFull{y} = [(x_{n} +
		y_{n})_{n \in \naturel}]_{\mathcal{U}}$
	\item $\hyperelementSequenceFull{x} ._{\hyperreal} \hyperelementSequenceFull{y} = [(x_{n} .
		y_{n})_{n \in \naturel}]_{\mathcal{U}}$
	\item $\hyperelementSequenceFull{x} <_{\hyperreal} \hyperelementSequenceFull{y} \equivalence
		\GSsetDef{n \in \naturel}{x_{n} < y_{n}} \in \mathcal{U}$
\end{enumerate}

On obtient évidemment une injection de $\real$ dans $\hyperreal$ avec
\begin{equation}
	i : \real \rightarrow \hyperreal : x \rightarrow [(x, x, \dots)]_{\mathcal{U}}
\end{equation}

où on prend la classe d'équivalence de la suite constante de valeur $x$. Nous
garderons la notation $\nonStandardVersion{x}$ pour désigner l'hyperréel défini
à partir du réel $x$. On dit que $\nonStandardVersion{x}$ est la \textbf{version
non standard} de $x$. Les éléments de la forme $\nonStandardVersion{x}$ sont
appelés les éléments \textbf{standards}. Nous définirons plus tard les versions
non standard des sous-ensembles de $\real$.

Ces deux opérations et l'ordre définies étendent les opérations réelles et
donnent une structure de corps ordonné sur $\hyperreal$.

\begin{proposition}
	$(\hyperreal, +_{\hyperreal}, ._{\hyperreal}, \nonStandardVersion{0},
	\nonStandardVersion{1},
<_{\hyperreal})$ est un corps ordonné.
\end{proposition}

\ifdefined\outputproof
\begin{proof}
	Par Los.
\end{proof}
\fi

\section{Eléments infinitésimaux et finis}

Maintenant que nous avons construit les hyperréels, nous allons distinguer trois
types d'éléments. Nous souhaitons définir \textit{être proche de zéro} ie
\textit{infinitésimal}, \textit{être proche d'un réel} et \textit{être infini}.

\begin{definition} [Infinitésimal]
	Soit $\hyperelement{x} \in \hyperreal$.

	On dit que $\hyperelement{x}$ est \textbf{infinitésimal} si pour tout $a \in
	\real^{> 0}$,
	\begin{equation}
		\nonStandardVersion{-a} \leq \hyperelement{x} \leq \nonStandardVersion{a}
	\end{equation}
\end{definition}

\begin{exemple}
	$\nonStandardVersion{0}$, $\hyperelementSequenceFullNoIndice{\frac{1}{n}}$ et
	$\hyperelementSequenceFullNoIndice{\frac{1}{\sqrt{n}}}$ sont des infinitésimaux.
\end{exemple}

\begin{definition} [Fini et infini]
	Soit $\hyperelement{x} \in \hyperreal$.

	On dit que $\hyperelement{x}$ est \textbf{fini} si il existe $a \in \real^{> 0}$,
	\begin{equation}
		\nonStandardVersion{-a} \leq \hyperelement{x} \leq \nonStandardVersion{a}
	\end{equation}

	L'ensemble des hyperréels finis est noté $\finiteSet{\hyperreal}$.

	Sinon, $\hyperelement{x}$ est dit \textbf{infini}.
\end{definition}

\begin{exemple}
	Toutes les versions non standard $\nonStandardVersion{x}$ d'un réel $x$ sont
	finies. Les hyperréels $\hyperelementSequenceFullNoIndice{n}$,
	$\hyperelementSequenceFullNoIndice{n^{2}}$ sont infinis.
\end{exemple}

\begin{proposition}
	Soient $\hyperelement{x}, \hyperelement{y} \in \hyperreal$. Alors

	\begin{enumerate}
		\item Si $\hyperelement{x}$ et $\hyperelement{y}$ sont infinitésimaux,
			alors $\hyperelement{x} + \hyperelement{y}$ est infinitésimal.
		\item Si $\hyperelement{x}$ et $\hyperelement{y}$ sont finis, alors
			$\hyperelement{x} + \hyperelement{y}$ est fini.
		\item Si $\hyperelement{x}$ est fini et $\hyperelement{y}$ est infini,
			alors $\hyperelement{x} \hyperelement{y}$ est infini.
	\end{enumerate}
\end{proposition}

\ifdefined\outputproof
\begin{proof}
	Evident.
\end{proof}
\fi

\begin{definition} [Infiniment proches]
	Soient $\hyperelement{x}$, $\hyperelement{y} \in \hyperreal$.

	On dit que $\hyperelement{x}$ et $\hyperelement{y}$ sont \textbf{infiniment
	proches} si $\hyperelement{x} - \hyperelement{y}$ est infinitésimal.

	On note alors $\hyperelement{x} \infinelyClosed \hyperelement{y}$.
\end{definition}

\begin{proposition}
	Soit $\hyperelement{x} \in \finiteSet{\hyperreal}$ un hyperréel fini.

	Alors, il existe un unique $a \in \real$ et un unique
	$\hyperelement{\epsilon} \in \hyperreal$
	infinitésimal tel que
	\begin{equation}
		\hyperelement{x} = \nonStandardVersion{a} +
		\hyperelement{\epsilon}
	\end{equation}

	En particulier, il existe un unique réel $a$ tel que $\hyperelement{x}
	\infinelyClosed \nonStandardVersion{a}$.
\end{proposition}

\ifdefined\outputproof
\begin{proof}

\end{proof}
\fi

\begin{definition} [Partie standard]
	Soit $\hyperelement{x} \in \hyperreal$ fini.

	L'unique réel $a$ tel que $\hyperelement{x} \infinelyClosed
	\nonStandardVersion{a}$ est appelé \textbf{partie standard de
		$\hyperelement{x}$} et est noté $\standardPartExpo{x}$ ou encore
		$\standardPart{x}$. Nous utiliserons la notation $\standardPart{x}$.
\end{definition}

\begin{notation}
	Pour $A \subseteq \hyperreal$ un sous-ensemble d'hyperréel fini, on note
	\begin{equation}
		\standardPart{A} = \GSsetDef{st(x)}{x \in A}
	\end{equation}
\end{notation}

\begin{proposition}
	La fonction
	\begin{equation}
		st : \finiteSet{\hyperreal} \rightarrow \real
	\end{equation}
	est
	\begin{enumerate}
		\item un morphisme d'anneau.
		\item une projection, c'est-à-dire $st \circ st = st$.
	\end{enumerate}
\end{proposition}

\ifdefined\outputproof
\begin{proof}

\end{proof}
\fi

Montrons quelques applications des infinitésimaux et plus généralement des
notions que nous avons vues. Nous allons montrer des résultats usuels d'analyse
grâce à l'analyse non standard.

\begin{proposition}
	Soit $\GSsequence{x}{n}{\naturel}$ une suite de réels convergeant vers un
	réel $x \in \real$.

	Alors, $\hyperelementSequence{x} \infinelyClosed \nonStandardVersion{x}$.
\end{proposition}

\ifdefined\outputproof
\begin{proof}
	Nous devons montrer que pour tout $a \in \real^{> 0}$, on a
	\begin{equation}
		\nonStandardVersion{x - a} < \hyperelementSequence{x} < \nonStandardVersion{x +
		a}
	\end{equation}
	Pour un $\epsilon > 0$ fixé, l'ensemble
	\begin{equation}
		\GSsetDef{n \in \naturel}{x - \epsilon < x_{n} < x + \epsilon}
	\end{equation}
	est cofini, donc est dans l'ultrafiltre $\mathcal{U}$.
\end{proof}
\fi

\section{Ensembles et fonctions internes}

\begin{definition} [Ensembles internes et externes]
	Soient $\GSsequence{A}{n}{\naturel}$ une suite de sous-ensembles de $\real$.

	On définit le sous ensemble hyperréel $\internalSet{A}$
	tel que
	\begin{equation}
		\hyperelementSequence{x} \in \internalSet{A} \equivalence \GSsetDef{n \in
		\naturel}{x_{n} \in A_{n}} \in \mathcal{U}
	\end{equation}

	En d'autre termes, pour appartenir à $\internalSet{A}$, il faut que les
	éléments $x_{n}$ apparaissent \textit{assez souvent} dans les fibres $A_{n}$.

	Les ensembles d'hyperréels pouvant être obtenus de cette façon sont appelés
	\textbf{ensembles internes}. Les autres sont appelés \textbf{externes}.

	On note l'ensemble des ensembles internes par $\setOfInternalSet{\hyperreal}$

	% Besoin d'étendre la notion aux suites d'ordinaux pour définir la
	% cardinalité \ldots
	%On définit alors \textbf{la cardinalité interne de $\internalSet{A}$ comme
	%l'élément \ldots
\end{definition}

\begin{exemple}
	Soient $\hyperelementSequence{a}$ et $\hyperelementSequence{b}$ deux hyperréels. Alors l'ensemble
	\begin{equation}
		[ \hyperelementSequence{a}, \hyperelementSequence{b} ] =
		\GSsetDef{\hyperelementSequence{x} \in
	\hyperreal}{\hyperelementSequence{a} \leq \hyperelementSequence{x} \leq
	\hyperelementSequence{b}}
	\end{equation}
	est interne et est obtenu en prenant $A_{n} = [a_{n}, b_{n}]$.
\end{exemple}

\begin{definition} [Fonctions internes]
	Soient $\GSsequence{f}{n}{\naturel}$ une suite de fonctions de $\real$ dans
	$\real$.

	On définit la fonction $\internalFunction{f} : \hyperreal \rightarrow
	\hyperreal$ tel que pour tous hyperréels $\hyperelementSequence{x}$
	\begin{equation}
		\internalFunction{f}(\hyperelementSequence{x}) =
		\hyperelementSequenceNoIndice{f_{n}(x_{n})}
	\end{equation}

	Les fonctions $f : \hyperreal \rightarrow \hyperreal$ pouvent être obtenues
	de cette façon, c'est-à-dire qu'il existe une suite
	$\GSsequence{f}{n}{\naturel}$ tel que pour tout $\hyperelementSequence{x}
	\in \hyperreal$,
	$f(\hyperelementSequence{x})$ est de la forme
	$\internalFunction{f}(\hyperelementSequence{x})$, sont appelées
	\textbf{fonctions internes}.
\end{definition}

\begin{proposition}
	Soient $\internalSet{A}$ et $\internalSet{B}$ deux ensembles internes,
	alors
	\begin{enumerate}
		\item $\internalSet{A} \inter \internalSet{B} = \internalSetNoIndice{A_{n}
			\inter B_{n}}$
		\item $\internalSet{A} \union \internalSet{B} = \internalSetNoIndice{A_{n}
			\union B_{n}}$
		\item $\comp{\internalSet{A}} = \internalSetNoIndice{\comp{A_{n}}}$
	\end{enumerate}

	De manière générale, cela est vrai pour n'importe quelle intersection et
	union finie.
\end{proposition}

\ifdefined\outputproof
\begin{proof}
	Cela découle des propriétés sur les ultrafiltres.
\end{proof}
\fi

\begin{proposition}
	Soit $(A^{i})_{i \in \naturel}$ une suite d'ensembles internes de
	$\hyperreal$.
	Chaque $A^{i}$ est de la forme $\internalSet{A^{i}}$.

	Si pour tous $i \in \naturel$,
	\begin{equation}
		A^{i} \neq \emptyset
	\end{equation}
	alors il existe $i \in \naturel$ tel que pour tous $n \in \naturel$,
	\begin{equation}
		A^{i}_{n} \neq \emptyset
	\end{equation}
\end{proposition}

\ifdefined\outputproof
\begin{proof}
	TODO
	%Si pour tous $i \in \naturel$, il existe $n \in \naturel$ tel que $A^{i}_{n
	%\in \naturel}$, on peut alors construire $\spanspace{A^{n}_{i}}$ qui est
	%vide car pour tous $i \in \naturel$, $A^{n}
\end{proof}
\fi

\begin{proposition}
	Soit $\internalSet{A} \subseteq \hyperreal$ un ensemble interne non vide.

	Si $\internalSet{A}$ admet un majorant alors $\internalSet{A}$ admet un supremum.
\end{proposition}

\begin{proof}
	Soit $\internalSet{A}$ borné par $\hyperelementSequence{a} \in \hyperreal$
	c'est-à-dire que $\internalSet{A} < \hyperelementSequence{a}$ ie
	\begin{equation}
		\overbrace{\GSsetDef{n \in \naturel}{A_{n} < a_{n}}}^{= I} \in \mathcal{U}
	\end{equation}
	Pour tous $i \in I$, on a $\sup(A_{i})$ qui existe.
	Posons $x_{i} = \sup(A_{i})$ si $i \in I$ et $0$ sinon.

	En particulier, $x_{i}$ est bien un majorant de $\internalSet{A}$ et c'est
	le plus petit. En effet, si $\hyperelementSequence{y}$ majore $\internalSet{A}$, notons $J$
	l'ensemble
	\begin{equation}
		\GSsetDef{n \in \naturel}{A_{n} < y_{n}} \in \mathcal{U}
	\end{equation}
	En particulier, on a $I \inter J \in \mathcal{U}$ et on a bien $[x_{i}] <
	[y_{i}]$ car
	\begin{equation}
		I \inter J \subseteq \GSsetDef{n \in \naturel}{x_{i} < y_{i}}
	\end{equation}
	D'où,
	\begin{equation}
		\GSsetDef{n \in \naturel}{x_{i} < y_{i}} \in \mathcal{U}
	\end{equation}
	Donc $\hyperelementSequence{x} < \hyperelementSequence{y}$.
\end{proof}

On vient donc de montrer que le principe du supremum reste valable pour les
\textit{ensembles internes} dans le cas hyperréels.

Ce principe permet de distinguer les ensembles internes et externes. Si un
ensemble $\internalSet{A}$ n'a pas de borne supérieure, alors c'est un
ensemble externe.

\begin{proposition}
	Soit $A \subseteq \hyperreal$ un ensemble interne.

	\begin{enumerate}
		\item \label{overflow} (overflow) Si $A$ contient des élément finis
			arbitrairement grands, alors $A$ contient un élément infini.
		\item \label{underflow} (underflow) Si $A$ contient des éléments
			positifs infinis arbitrairement petits, alors $A$ contient un
			élément fini.
	\end{enumerate}
\end{proposition}

\ifdefined\outputproof
\begin{proof}
	\begin{enumerate}
		\item
		\item
	\end{enumerate}
\end{proof}
\fi

\begin{theorem} [$\aleph_{1}$-saturation]
	\label{theorem:aleph_1_saturation}
	Soit $(A^{i})_{i \in \naturel}$ une suite d'ensembles internes de
	$\hyperreal$ tel que pour tous $I \in \naturel$
	\begin{equation}
		\inter_{i \leq I} A^{i} \neq \emptyset
	\end{equation}
	alors
	\begin{equation}
		\inter_{n \in \naturel} A^{n} \neq \emptyset
	\end{equation}
\end{theorem}

Remarquons que cela n'est pas nécessairement vrai que cela reviendrait à dire
que l'intersection dénombrable est stable dans un ultrafiltre, ce qui n'est pas
nécéssairement vrai.

\ifdefined\outputproof
\begin{proof}
	Les $A^{i}$ sont de la forme $\spanspace{A^{i}_{n}}$. Comme
	\begin{equation}
		\inter_{i \leq I} A^{i} \neq \emptyset
	\end{equation}
	pour tous $I \in \naturel$, les $A^{i} \neq
	\emptyset$. En particulier, on peut supposer que $A^{0}_{n}$ est non vide
	pour tous $n \in \naturel$.

	On a, pour tous $I \in \naturel$,
	\begin{equation}
		\inter_{i \leq I} \spanspace{A^{i}_{n}} = \spanspace{\inter_{i \leq I}
		A^{i}_{n}}
	\end{equation}
	ce qui implique
	\begin{equation}
		\GSsetDef{n \in \naturel}{\inter_{i \in I} A^{i}_{n} \neq \emptyset} \in
		\mathcal{U}
	\end{equation}

	Soit $n \in \naturel$. Posons
	\begin{equation}
		I_{n} = \max \overbrace{\GSsetDef{I \in \naturel}{\inter_{i \leq I}
		A^{i}_{n}} \neq \emptyset \wedge I \leq n}^{= X_{n}}
	\end{equation}

	Comme on a supposé que $A^{0}_{n}$ est non vide pour tous $n \in \naturel$,
	on a $I_{n} \geq 0$ car l'ensemble est non vide ($0$ appartient pour tous $n
	\in \naturel$) et le max existe car $I \leq n$.

	Pour $n$ fixé, on prend un élément $x_{n} \in \inter_{i \in I_{n}}
	A^{i}_{n}$. On construit ainsi une suite $(x_{n})_{n \in \naturel}$.

	Montrons maintenant que $[x_{n}] \in A^{I}$ pour tous $I \in \naturel$. Pour
	un $I \in \naturel$ fixé, on a
	\begin{equation}
		\GSsetDef{n \in \naturel}{I \leq I_{n}} \subseteq \GSsetDef{n \in
		\naturel}{x_{n} \in A^{I}_{n}}
	\end{equation}
	car si $I \leq I_{n}$, on a $I \in X_{n}$ et par conséquence, comme $x_{n}
	\in \inter_{i \in I_{n}} A^{i}_{n}$ et $\inter_{i \in I} A^{i}_{n} \subseteq \inter_{i
		\in I_{n}} A^{i}_{n}$, $n \in \GSsetDef{m \in \naturel}{x_{m} \in
		A^{I}_{m}}$.

	De plus, par définition de $I_{n}$.
	\begin{equation}
		\GSsetDef{n \in \naturel}{I \leq I_{n}} = \GSsetDef{n \in \naturel}{I
		\leq n} \inter \GSsetDef{n \in \naturel}{\inter_{i \leq I} A^{i}_{n}
	\neq \emptyset}
	\end{equation}
	Comme $\GSsetDef{n \in \naturel}{I \leq n}$ est cofini, il appartient à
	$\mathcal{U}$. D'où $\GSsetDef{n \in \naturel}{I \leq I_{n}} \in
	\mathcal{U}$ et par conséquence,
	\begin{equation}
		\GSsetDef{n \in \naturel}{x_{n} \in A^{I}_{n}} \in \mathcal{U}
	\end{equation}
	ie $[x_{n}] \in A^{I}$.
\end{proof}
\fi

\begin{corollary}
	Soit $\spanspace{A^{i}}_{i \in \naturel}$ une suite d'ensembles internes de
	$\hyperreal$.
	Alors, les assertions suivantes sont équivalentes
	\begin{enumerate}
		\item $\union_{i \in \naturel} A^{i}$ est interne
		\item il existe $n \in \naturel$ tel que
			\begin{equation}
				\union_{i \in \naturel} A^{i} = \union_{i \leq n} A^{i}
			\end{equation}
	\end{enumerate}
\end{corollary}

\ifdefined\outputproof
\begin{proof}
	$(2) \implies (1)$ est évident.

	$(1) \implies (2)$ Supposons que $\union_{n \in \naturel A_{n}}$ est
	interne et posons $A = \union_{n \in \naturel A_{n}}$. Alors, pour tout $n
	\in \naturel$, $A/A_{n}$ est aussi interne.
	De plus,
	\begin{equation}
		\inter_{n \in \naturel} A/A_{n} = \emptyset
	\end{equation}
	Donc, par la contraposée de la $\aleph_{1}$-saturation
	(\ref{theorem:aleph_1_saturation}), il existe $N \in \naturel$ tel que
	\begin{equation}
		\inter_{n \leq N} A/A_{n} = \emptyset
	\end{equation}
	c'est-à-dire
	\begin{equation}
		A = \union_{n \leq N} A_{n}
	\end{equation}
\end{proof}
\fi

\begin{proposition}
	Soit $A \subseteq \hyperreal$ un sous-ensemble interne.

	Alors, $st(A)$ est fermé dans $(\real, \tau_{\abs{.}})$.
\end{proposition}

\ifdefined\outputproof
\begin{proof}

\end{proof}
\fi

\section{Ensembles standards et hyperfinis}

Pour les hyperréels $\hyperelement{x}$, on a distingué ceux qui étaient finis. On a alors
défini une fonction $st : Fin(\hyperreal) \rightarrow \real$ qu'on a appelé la
partie standard d'un élément fini.
Nous allons définir la même chose, mais cette fois-ci avec les fonctions et les
ensembles.

\begin{definition}
	Soit $A \subseteq \real$ un sous-ensemble réel. On définit \textbf{la
	version non standard de $A$} comme l'ensemble interne $\nonStandardVersion{A} =
	\internalSet{A}$ où pour tous $n \in \naturel$, $A_{n} = A$.

	On dit qu'un sous-ensemble de $\hyperreal$ est \textbf{standard} s'il
	est de la forme $\nonStandardVersion{A}$.
\end{definition}

\begin{definition}
	Soit $f : \real \rightarrow \real$ une fonction. On définit \textbf{la
	version non standard de $f$} comme la fonction interne
	\begin{equation}
		\nonStandardVersion{f} = \internalFunction{f} : \hyperreal \rightarrow \hyperreal
	\end{equation}

	où pour tous $n \in \naturel$, $f_{n} = f$.

	On dit qu'une fonction interne $\internalFunction{f}$ est \textbf{standard}
	si elle est de la forme $\nonStandardVersion{f}$.
\end{definition}

%\begin{remarque}
	%Pour un ensemble $A \subseteq \real$, sa version non standard
	%$\nonStandardVersion{A}$ contient bien plus d'élément que \textit{simplement} que
	%$\nonStandardVersion{A}$.
%\end{remarque}

%\begin{proposition}
	%Soit $A \subseteq \real$.

	%$A$ est fini si et seulement si $\nonStandardVersion{A} = \nonStandardVersion{A}$.
%\end{proposition}

%\ifdefined\outputproof
%\begin{proof}

%\end{proof}
%\fi

Remarquons que la version non standard $\nonStandardVersion{f}$ d'une fonction $f :
\real \rightarrow \real$ étend la fonction $f$ dans le sens que pour tous $a \in
\real$:

\begin{equation}
	\nonStandardVersion{f}(\hyperelement{a}) =
	\internalFunctionNoIndice{f}(\hyperelement{a}) = \hyperelement{f(a)} = i(f(a))
\end{equation}

On peut aussi former la version non standard des ensembles usuels. On construit
ainsi
\begin{enumerate}
	\item $\nonStandardVersion{\naturel}$, noté $\hyperstructure{\naturel}$: les
		naturels non standards, à partir de $\naturel$.
	\item $\nonStandardVersion{\integer}$, noté $\hyperstructure{\integer}$: les
		entiers non standards, à partir de $\integer$.
	\item $\nonStandardVersion{\rational}$, noté $\hyperstructure{\rational}$:
		les rationnels non standards, à partir de $\rational$.
\end{enumerate}

Les naturels non standards sont par exemples les classes d'équivalences de
suites
$\GSsequence{N}{n}{\naturel}$ où $\GSsetDef{n \in \naturel}{N_{n} \in \naturel}
\in \mathcal{U}$, c'est-à-dire les classes d'équivalences des suites dont
presque tous les éléments appartiennent à $\naturel$.

Par exemple, $[(\frac{1}{2}, \frac{1}{4}, 1, 2, 3, \dots)]_{\mathcal{U}}$ est un
naturel non standard.

Nous allons continuer avec les ensembles hyperfinis.

\begin{definition} [Ensemble hyperfini et cardinalité hyperfinie]
	Soit un ensemble interne $\internalSet{A} \subseteq \hyperreal$. On dit
	que $\internalSet{A}$ est \textbf{un ensemble hyperfini} si
	\begin{equation}
		\GSsetDef{n \in \naturel}{A_{n} \text{ est fini}} \in \mathcal{U}
	\end{equation}

	La cardinalité d'un ensemble hyperfini $\internalSet{A}$ est le naturel
	non standard $\nonStandardVersion{\cardinal{A_{n}}}$ où $\cardinal{A_{n}}$ est la
	cardinalité de $A_{n}$.
\end{definition}

\begin{remarque}
	La cardinalité d'un ensemble hyperfini $\internalSet{A}$ est bien un
	naturel non standard car l'ensemble des naturels décrivant les $A_{n}$ est
	dans l'ultrafiltre, d'où la classe d'équivalence de la suite des cardinaux
	est bien dans l'ultrafiltre, c'est-à-dire dans les naturels non standards.
\end{remarque}

\begin{exemple}
	Soit $n \in \naturel$ et construisons l'élément $[n] \in \naturel$ à partir
	de la suite constante de naturels de constante $n$, ie $(n, n, n, n, n,
	\cdots)$.

	Posons $A_{n} = \GSset{0, \frac{1}{n}, \frac{2}{n}, \cdots, \frac{n -
	1}{n}, 1} \subseteq \naturel$ et construisons l'ensemble interne
	$\internalSet{A}$.

	Alors, pour tous $n \in \naturel$, $A_{n}$ est fini, de cardinalité $n + 1$.
	Donc $\internalSet{A}$ est hyperfini de cardinalité $[n + 1]$.
\end{exemple}

Remarquons que les ensembles hyperfinis héritent de différentes propriétés des
ensembles finis. Par exemple
\begin{proposition}
	Soient $\internalSet{A}$ et $\internalSet{B}$ un ensemble hyperfini. Alors
	\begin{enumerate}
		\item $\internalSet{A}$ et $\internalSet{B}$ possèdent un plus petit élément.
		\item $\internalSet{A}$ et $\internalSet{B}$ possèdent un plus grand élément.
		\item $\internalSet{A} \union \internalSet{B}$ est hyperfini.
	\end{enumerate}
\end{proposition}

\begin{proof}
	Laissée au lecteur.
\end{proof}

D'autres notions peuvent être étendu aux ensembles hyperfinis à partir des
ensembles finis.

Prenons un ensemble hyperfini $A = \internalSet{A}$ et une fonction interne
$f = \internalFunction{f}$. On peut définir la somme des images de $f$ sur $A$
comme
\footnote{Cela a bien du sens que $A_{n} \subseteq \real$ et
$f_{n} : \real \rightarrow \real$}
\begin{equation}
	\sum_{a \in A} f(a) = \hyperelementSequenceNoIndice{\sum_{a_{n} \in A_{n}} f_{n}(a_{n})}
\end{equation}


% Besoin de définir bijections internes et de cardinalité interne.

%Suites aux propriétés ci-dessus, on est amené à une définition équivalente
%d'ensemble hyperfini.

%\begin{proposition}
	%Soit $\internalSet{A}$ un ensemble interne. Les assertions suivantes sont
	%équiavlentes:
	%\begin{enumerate}
		%\item $\internalSet{A}$ est un ensemble hyperfini de cardinalité $[N]$
	%\end{enumerate}
%\end{proposition}

\section{Calcul infinitésimal}

Maintenant que nous avons défini la notion d'infinitésimaux, il est intéressant
de faire l'analogie avec le calcul infinitésimal. Nous allons voir que la notion
d'infinitésimaux n'est pas anodine. En fait, les hyperréels ont été développés
pour donner une définition précise d'\textit{infinitésimaux}.

\begin{proposition}
	Soit $f : \real \rightarrow \real$ et $a \in \real$. Les assertions
	suivantes sont équivalentes:
	\begin{enumerate}
		\item $f$ est continue en $a$
		\item $\nonStandardVersion{f}(x) \infinelyClosed
			\nonStandardVersion{f}(\nonStandardVersion{a})$ pour tous $x
			\in \hyperreal$ tel que $x \infinelyClosed \nonStandardVersion{a}$.
	\end{enumerate}
\end{proposition}

\ifdefined\outputproof
\begin{proof}

\end{proof}
\fi

La continuité de la composition de deux fonctions continues devient alors
évidente.

\begin{proposition}
	Soient $f, g : \real \rightarrow \real$. Soit $a \in \real$ tel que $g$ est
	continue en $a$ et $f$ est continue en $g(a)$.

	Alors $f \circ g$ est continue en $a$.
\end{proposition}

\ifdefined\outputproof
\begin{proof}
	Si $x \infinelyClosed a$, alors $\nonStandardVersion{g}(x) \infinelyClosed \nonStandardVersion{g}(a)$ ($g$
	continue en $a$) et
	cela induit que $\nonStandardVersion{f}(\nonStandardVersion{g}(x))
	\infinelyClosed \nonStandardVersion{f}(\nonStandardVersion{g}(a))$ car $f$ est continue en $g(a)$.

	On a donc bien montré que $f \circ g$ est continue en $a$ car pour tous $x \infinelyClosed
	a$, $(\nonStandardVersion{f} \circ \nonStandardVersion{g}) (x) \infinelyClosed (\nonStandardVersion{f}
	\circ \nonStandardVersion{g})(a)$.
\end{proof}
\fi

La propriété qu'une fonction continue sur un compact atteint ses bornes est
aussi évident:

\begin{proposition}
	Soit $f : K \rightarrow \real$ une fonction continue et $K$ un compact de
	$\real$.

	Alors $f$ atteint ses bornes.
\end{proposition}

\ifdefined\outputproof
\begin{proof}

\end{proof}
\fi

On peut aussi donner une équivalence pour la notion de continuité uniforme.

\begin{proposition}
	Soit $f : \real \rightarrow \real$ et $A \subseteq \real$. Les assertions
	suivantes sont équivalentes:

	\begin{enumerate}
		\item $f$ est uniformément continue sur $A$.
		\item $\nonStandardVersion{f}(x) \infinelyClosed \nonStandardVersion{f}(y)$ pour tous éléments
			non standards $x, y \in \nonStandardVersion{A}$.
	\end{enumerate}
\end{proposition}

\ifdefined\outputproof
\begin{proof}

\end{proof}
\fi

On obtient alors une preuve simple de la proposition suivante:

\begin{proposition}
	Soit $f : K \rightarrow \real$ une fonction continue et $K$ un interval compact.
	Alors $f$ est uniformément continue sur $K$.
\end{proposition}

\ifdefined\outputproof
\begin{proof}

\end{proof}
\fi

La définition de dérivée a également une équivalence hyperréelle.

\begin{proposition}
	Soit $f : \real \rightarrow \real$ et $a \in \real$. Les assertions
	suivantes sont équivalentes:

	\begin{enumerate}
		\item $f$ est dérivable en $a$
		\item il existe $b \in \real$ tel que pour tous $x \in \hyperreal$
			\begin{equation}
				x \infinelyClosed [a] \implies \frac{\nonStandardVersion{f}(x) -
				\nonStandardVersion{f}([a])}{x - [a]} \infinelyClosed b
			\end{equation}
			Si ce $b$ existe, alors, $b = f'(a)$.
	\end{enumerate}
\end{proposition}

\ifdefined\outputproof
\begin{proof}

\end{proof}
\fi

% Chain Rule et solution équation différentielle.

