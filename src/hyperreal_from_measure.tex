\chapter{Construction des hyperréels et des superstructures grace à une mesure}

Pour le reste du document, nous fixons une mesure additive finie $m$ répondant
aux critères précédent.

\begin{notation}
	Soit $\real$ l'ensemble des nombres réels. On note $\mathcal{S}$ l'ensemble
	des suites réelles.
\end{notation}

\begin{definition}
	On définit la relation $\approx$ sur l'ensemble $\mathcal{S}$ par
	\begin{equation}
		(a_{n}) \approx (b_{n}) \equiv m(\GSsetDef{n \in \naturel}{a_{n} = b_{n}})
	\end{equation}
\end{definition}

\begin{proposition}
	La relation $\approx$ définie précédemment est une relation d'équivalence sur
	$\mathcal{S}$.
\end{proposition}

\ifdefined\outputproof
\begin{proof}

\end{proof}
\fi

\begin{definition}
	Soit $\mathcal{S}$ l'ensemble des suites réelles et $\approx$ la relation
	d'équivalence définie précédemment.

	On appelle l'ensemble quotient $\mathcal{S} / \approx$ \textbf{l'ensemble des hyperréels} ou
	\textbf{l'ensemble des réels non standards}.
\end{definition}

Construisons maintenant des opérations sur $\hyperreal$.

\begin{definition}
	Soit $\hyperreal$ l'ensemble des hyperréels et soient $[(a_{n})_{n \in
	\naturel}]$ et $[(b_{n})_{n \in \naturel}]$ deux hyperréels.

	On définit l'opération $+_{\hyperreal}$ tel que
	\begin{equation}
		[(a_{n})_{n \in \naturel}] +_{\hyperreal} [(b_{n})_{n \in \naturel}] = [(a_{n} +
		b_{n})_{n \in \naturel}]
	\end{equation}
	et l'opération $._{\hyperreal}$ tel que
	\begin{equation}
		[(a_{n})_{n \in \naturel}] ._{\hyperreal}  [(b_{n})_{n \in \naturel}] =
		[(a_{n} .
		b_{n})_{n \in \naturel}]
	\end{equation}

	Par abus de notations, on omettra les indices sur les opérations.
\end{definition}

\begin{definition}
	Soit $\hyperreal$ l'ensemble des hyperréels et soient $[(a_{n})_{n \in
	\naturel}]$ et $[(b_{n})_{n \in \naturel}]$ deux hyperréels.

	On définit la relation d'ordre $<_{\hyperreal}$ tel que
	\begin{equation}
		[(a_{n})_{n \in \naturel}] <_{\hyperreal} [(b_{n})_{n \in \naturel}]
	\end{equation}
	ssi
	\begin{equation}
		m(\GSsetDef{n \in \naturel}{a_{n} < b_{n}}) = 1
	\end{equation}
\end{definition}

Remarquons que ces deux opérations et l'ordre défini permettent de retrouver des
caractéristiques de $\real$.

\begin{proposition}
	Soit $\hyperreal$ l'ensemble des hyperréels, $+$ et $.$ les deux opérations
	d'addition et de multiplication définies précédemment.
	Soient $0_{\hyperreal} = [0]$, $1_{\hyperreal} = [1]$ et $<_{\hyperreal}$
	l'ordre défini précédemment sur $\hyperreal$.

	Alors $(\hyperreal, +, ., 0_{\hyperreal}, <_{\hyperreal})$ forme un corps
	ordonné.
\end{proposition}

\ifdefined\outputproof
\begin{proof}

\end{proof}
\fi
